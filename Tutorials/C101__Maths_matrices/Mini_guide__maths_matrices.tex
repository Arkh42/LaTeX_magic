%             %
%%           %%
%%% CHAPTER %%%
%%           %%
%             %


\chapter{Mathematics -- Matrices}

	\label{cha::maths:matrix}
	

Engineers and scientists use many matrices to describe physical phenomena.
Hence the frequently asked question: \enquote{how can I write a matrix with \LaTeX{}}?
This chapter answer the question.



\section{Packages for matrices}


	As for the mathematics basics (cf. \cref{cha::maths:basics}), the \emph{mathtools} package, which is an upgrade of the well-known and widely used \emph{amsmath} package, provides everything that is necessary to write matrix computations.
	
	In addition, the \emph{physics} package provides very useful commands to facilitate the creation of specific matrices such as an identity matrix or a zero matrix.
	
	So the following lines should be written in the preamble of the document:
\begin{lstlisting}[language={[LaTeX]TeX}]
\usepackage{mathtools}
\usepackage{physics}
\end{lstlisting}



\section{Writing matrices}


	Matrices can be generated thanks to the \texttt{matrix} environment which must be used inside a mathematical equation environment.
	The simplest \LaTeX{} code which generates a matrix is
\begin{lstlisting}[language={[LaTeX]TeX}]
\begin{equation*}
	\begin{matrix}
		x_{11} & x_{12} \\
		x_{21} & x_{22}
	\end{matrix} 
\end{equation*}
\end{lstlisting}
	and results in
	\begin{equation*}
		\begin{matrix}
			x_{11} & x_{12} \\
			x_{21} & x_{22}
		\end{matrix}
	\end{equation*}
	
	
	There are several variants of \texttt{matrix} which produce different delimiters surrounding the matrix.
	They are presented here below, with the corresponding environment's name on top of each matrix:
	\begin{align*}
		\text{\texttt{matrix}} && \text{\texttt{pmatrix}} && \text{\texttt{bmatrix}} && \text{\texttt{Bmatrix}} \\
		%
		\begin{matrix}
			x_{11} & x_{12} \\
			x_{21} & x_{22}
		\end{matrix} 
		&&
		\begin{pmatrix}
			x_{11} & x_{12} \\
			x_{21} & x_{22}
		\end{pmatrix}
		&&
		\begin{bmatrix}
			x_{11} & x_{12} \\
			x_{21} & x_{22}
		\end{bmatrix}
		&&
		\begin{Bmatrix}
			x_{11} & x_{12} \\
			x_{21} & x_{22}
		\end{Bmatrix} \\
		%
		%
		&& \text{\texttt{vmatrix}} && \text{\texttt{Vmatrix}} && \\
		%
		&&
		\begin{vmatrix}
			x_{11} & x_{12} \\
			x_{21} & x_{22}
		\end{vmatrix}
		&&
		\begin{Vmatrix}
			x_{11} & x_{12} \\
			x_{21} & x_{22}
		\end{Vmatrix}
		&& 
	\end{align*}
	
	
	The \emph{mathtools} package offers starred versions of the \texttt{matrix} environments which allow to pass an optional argument to specify the alignment within the matrix's columns.
	A \LaTeX{} example is shown here below:
\begin{lstlisting}[language={[LaTeX]TeX}]
\begin{align*}
	\begin{pmatrix}
		2  & -3 \\
		42 & 0
	\end{pmatrix}	&& \text{VS} &&
	\begin{pmatrix*}[r]
		2  & -3 \\
		42 & 0
	\end{pmatrix*}
\end{align*}.
\end{lstlisting}
	Observe the difference in alignment between \texttt{pmatrix} (left) and \texttt{pmatrix*} (right):
	\begin{align*}
		\begin{pmatrix}
		2  & -3 \\
		42 & 0
		\end{pmatrix}	&& \text{VS} &&
		\begin{pmatrix*}[r]
		2  & -3 \\
		42 & 0
		\end{pmatrix*}.
	\end{align*}



\section{Facilities for specific matrices}


	The \emph{physics} package proposes many commands that make the \LaTeX{}'s life of scientists and engineers easier.
	Among them, there are commands aiming the easy generation of specific matrices.
	
	
	
	\subsection{Matrices with specific patterns}
	
		
		In mathematics, there are specific matrices that are commonly used.
		These matrices are specific because they matches a pattern that is immediately recognised.
		The most known are the zero matrix, the identity matrix and the diagonal matrix.
		
		The \emph{physics} package provides commands to automatically generate the desired pattern for an $n \times m$ matrix, $n$ and $m$ being adapted according to the arguments passed to the commands.
		All commands must be placed inside a \texttt{matrix}-like environment, itself place within an \texttt{equation}-like environment.
		
		
		
		\subsubsection{Zero matrix}
		
		
			The zero matrix is generated with the \texttt{zeromatrix} command, or the shorter form \texttt{zmat}.
			The first argument is the dimension $n$, second argument is the dimension $m$ of the $n \times m$ matrix.
			If $m$ is omitted, the command generates a square zero matrix.
			
			A comparison of the equivalent \LaTeX{} codes
\begin{lstlisting}[language={[LaTeX]TeX}]
\begin{align*}
	% 3x3 matrix
	\begin{pmatrix} % based on physics
		\zmat{3}
	\end{pmatrix}	&&
	\begin{pmatrix} % regular way
		0 & 0 & 0 \\
		0 & 0 & 0 \\
		0 & 0 & 0
	\end{pmatrix}	&& \text{and} &&
	% 3x5 matrix
	\begin{pmatrix} % based on physics
		\zmat{3}{5}
	\end{pmatrix}	&&
	\begin{pmatrix} % regular way
		0 & 0 & 0 & 0 & 0 \\
		0 & 0 & 0 & 0 & 0 \\
		0 & 0 & 0 & 0 & 0
	\end{pmatrix} 
\end{align*}
\end{lstlisting}
			generating
			\begin{align*}
				% 3x3 matrix
				\begin{pmatrix} % based on physics
					\zmat{3}
				\end{pmatrix}	&&
				\begin{pmatrix} % regular way
					0 & 0 & 0 \\
					0 & 0 & 0 \\
					0 & 0 & 0
				\end{pmatrix}	&& \text{and} &&
				% 3x5 matrix
				\begin{pmatrix} % based on physics
					\zmat{3}{5}
				\end{pmatrix}	&&
				\begin{pmatrix} % regular way
					0 & 0 & 0 & 0 & 0 \\
					0 & 0 & 0 & 0 & 0 \\
					0 & 0 & 0 & 0 & 0
				\end{pmatrix}.
			\end{align*}
			shows that using the \emph{physics} package avoids a fastidious copy-paste work.
		
		
		
		\subsubsection{Identity matrix}
		
		
			The identity matrix is generated with the \texttt{identitymatrix} command, or the shorter form \texttt{imat}.
			The argument is the dimension $n$ of the square matrix.
			
			
			Comparing the \enquote{regular} \LaTeX{} with the code based on \emph{physics}
\begin{lstlisting}[language={[LaTeX]TeX}]
\begin{equation*}
	\begin{pmatrix} % based on physics
		\imat{3}
	\end{pmatrix}	\quad
	\begin{pmatrix} % regular way
		1 & 0 & 0 \\
		0 & 1 & 0 \\
		0 & 0 & 1
	\end{pmatrix}
\end{equation*}
\end{lstlisting}
			while both result in the same display
			\begin{equation*}
				\begin{pmatrix} % based on physics
					\imat{3}
				\end{pmatrix}	\quad
				\begin{pmatrix} % regular way
					1 & 0 & 0 \\
					0 & 1 & 0 \\
					0 & 0 & 1
				\end{pmatrix}
			\end{equation*}
			highligthts the time saved in this case, especially because the copy-paste is not as efficient as with the zero matrix.
			
		
		
		\subsubsection{Diagonal matrix}
		
			
			A diagonal matrix is generated with the \texttt{diagonalmatrix} command, or the shorter form \texttt{dmat}.
			The argument is a list of comma-separated values, the first one being $x_{11}$, the second one being $x_{22}$, and so forth.
			By default, only the values on the diagonal are printed and the rest is filled with white spaces.
			An optional argument allows to fill the spaces.
			
			For instance, the following code
\begin{lstlisting}[language={[LaTeX]TeX}]
\begin{align*}
	% based on physics
	\begin{pmatrix}
		\dmat{a,b,c}
	\end{pmatrix}	&&
	\begin{pmatrix}
		\dmat{1,2,3}
	\end{pmatrix}	&&
	\begin{pmatrix}
		\dmat[0]{1,2,3}
	\end{pmatrix}	\\
	% regular way
	\begin{pmatrix}
		a &   &   \\
		  & b &   \\
		  &   & c
	\end{pmatrix}	&&
	\begin{pmatrix}
		1 &   &   \\
		  & 2 &   \\
		  &   & 3
	\end{pmatrix}	&&
	\begin{pmatrix}
		1 & 0 & 0 \\
		0 & 2 & 0 \\
		0 & 0 & 3
	\end{pmatrix}
\end{align*}
\end{lstlisting}
			generates
			\begin{align*}
				% based on physics
				\begin{pmatrix}
					\dmat{a,b,c}
				\end{pmatrix}	&&
				\begin{pmatrix}
					\dmat{1,2,3}
				\end{pmatrix}	&&
				\begin{pmatrix}
					\dmat[0]{1,2,3}
				\end{pmatrix},	\\
				% regular way
				\begin{pmatrix}
					a &   &   \\
					  & b &   \\
					  &   & c
				\end{pmatrix}	&&
				\begin{pmatrix}
					1 &   &   \\
					  & 2 &   \\
					  &   & 3
				\end{pmatrix}	&&
				\begin{pmatrix}
					1 & 0 & 0 \\
					0 & 2 & 0 \\
					0 & 0 & 3
				\end{pmatrix}.
			\end{align*}
			The above example also shows that literal values can be used as well as numerical values.
			
			
			Similarly to the diagonal matrix, an anti-diagonal matrix can be created with the \texttt{antidiagonalmatrix} command, or the shorter form \texttt{admat}.
		
		
		
		\subsubsection{Generalised pattern}
		
		
			An $n \times m$ matrix can be filled with a specific element thanks to the \texttt{xmatrix} command, or the shorter form \texttt{xmat}.
			A starred version adds automatic indices.
			
			As an example, the
\begin{lstlisting}[language={[LaTeX]TeX}]
\begin{align*}
	% based on physics
	\begin{pmatrix}
		\xmat{1}{2}{3}
	\end{pmatrix}	&&
	\begin{pmatrix}
		\xmat*{x}{3}{3}
	\end{pmatrix}	&&
	\begin{pmatrix}
		\xmat*{a}{3}{1}
	\end{pmatrix}	&&
	\begin{pmatrix}
		\xmat*{\alpha}{1}{3}
	\end{pmatrix}	\\
	% regular way
	\begin{pmatrix}
		1 & 1 & 1 \\
		1 & 1 & 1
	\end{pmatrix}	&&
	\begin{pmatrix}
		x_{11} & x_{12} & x_{13} \\
		x_{21} & x_{22} & x_{23} \\
		x_{31} & x_{32} & x_{33}
	\end{pmatrix}	&&
	\begin{pmatrix}
		a_{1} \\
		a_{2} \\
		a_{3}
	\end{pmatrix}	&&
	\begin{pmatrix}
		\alpha_{1} & \alpha_{2} & \alpha_{3}
	\end{pmatrix}
\end{align*}
\end{lstlisting}
			\LaTeX{} codes produces
			\begin{align*}
				% based on physics
				\begin{pmatrix}
					\xmat{1}{2}{3}
				\end{pmatrix}	&&
				\begin{pmatrix}
					\xmat*{x}{3}{3}
				\end{pmatrix}	&&
				\begin{pmatrix}
					\xmat*{a}{3}{1}
				\end{pmatrix}	&&
				\begin{pmatrix}
					\xmat*{\alpha}{1}{3}
				\end{pmatrix},	\\
				% regular way
				\begin{pmatrix}
					1 & 1 & 1 \\
					1 & 1 & 1
				\end{pmatrix}	&&
				\begin{pmatrix}
					x_{11} & x_{12} & x_{13} \\
					x_{21} & x_{22} & x_{23} \\
					x_{31} & x_{32} & x_{33}
				\end{pmatrix}	&&
				\begin{pmatrix}
					a_{1} \\
					a_{2} \\
					a_{3}
				\end{pmatrix}	&&
				\begin{pmatrix}
					\alpha_{1} & \alpha_{2} & \alpha_{3}
				\end{pmatrix}.
			\end{align*}
			As observed, the automatic indexing takes the number of lines and columns into account.
			Automatic indexing is especially useful for general formulae and demonstrations.
			Finally, like for the diagonal matrix, literal (including Greek letters) values can be used.
			
			At this point, the author believes that the reader is convinced by the power of the \emph{physics} package to create matrices which match a specific pattern.
			
	
	
	\subsection{Combinations of patterns}
	
	
		Even though the aforementioned commands are usually sufficient for most cases, it is sometimes required to combine patterns together to obtain the desired result.
		The \emph{physics} allow to do so \footnote{This section can be considered as intended for \enquote{advanced} users. However, the author thinks that separate it from the rest of the explanations about matrices is non-sense.}.
		
		
		A first possibility consists in creating the pattern and then adding the other elements around.
		The
\begin{lstlisting}[language={[LaTeX]TeX}]
\begin{equation*}
	\begin{pmatrix}
		\imat{2} \\
		a & b
	\end{pmatrix}	\quad
	\begin{pmatrix}
		a & b & c \\
		\zmat{2}{3}
	\end{pmatrix}
\end{equation*}
\end{lstlisting}
		example code outputs
		\begin{equation*}
			\begin{pmatrix}
				\imat{2} \\
				a & b
			\end{pmatrix}	\quad
			\begin{pmatrix}
				a & b & c \\
				\zmat{2}{3}
			\end{pmatrix}.
		\end{equation*}
		
		However, if an element is added on the right or on the left of a pattern, the result is not the one expected, as depicted by the following code:
\begin{lstlisting}[language={[LaTeX]TeX}]
\begin{equation*}
	\begin{pmatrix}
		\imat{2} & a \\
		         & b
	\end{pmatrix}	\quad
	\begin{pmatrix}
		         & a \\
		\imat{2} & b
	\end{pmatrix}	\quad
	\begin{pmatrix}
		a & \\
		b & \zmat{2}{3}
	\end{pmatrix}.
\end{equation*}
\end{lstlisting}
		\begin{equation*}
			\begin{pmatrix}
				\imat{2} & a \\
				         & b
			\end{pmatrix}	\quad
			\begin{pmatrix}
						 & a \\
				\imat{2} & b
			\end{pmatrix}	\quad
			\begin{pmatrix}
				a & \\
				b & \zmat{2}{3}
			\end{pmatrix}.
		\end{equation*}
		
		
		To overcome this issue, the \texttt{matrixquantity} command, or the shorter form \texttt{mqty}, can be used.
		It creates a matrix as a single element.
		
		Here follows an example that is explained below:
\begin{lstlisting}[language={[LaTeX]TeX}]
\begin{equation*}
	\begin{pmatrix}
		\mqty{\imat{2}} & \mqty{a\\b} \\
		\mqty{c & d}    & e
	\end{pmatrix}
\end{equation*}
\end{lstlisting}
		\begin{equation*}
			\begin{pmatrix}
				\mqty{\imat{2}} & \mqty{a\\b} \\
				\mqty{c & d}    & e
			\end{pmatrix}.
		\end{equation*}
		Firstly, the identity matrix is created and embedded in \texttt{mqty} to be considered as a single element.
		Secondly, we can add a $2 \times 1$ vector on the right but he must also be encapsulated in \texttt{mqty} as the identity matrix is a single element.
		At this step, \LaTeX{} sees a $1 \times 2$ matrix, even though the displayed result is a $2 \times 3$ matrix.
		Hence $c$ and $d$ that are embedded in \texttt{mqty} whereas $e$ is a \enquote{regular} single element.
		

% *** End of chapter ***