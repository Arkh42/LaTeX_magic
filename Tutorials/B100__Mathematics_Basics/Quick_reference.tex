%              %
%%            %%
%%% PREAMBLE %%%
%%            %%
%              %

% Document class
\documentclass[11pt]{beamer}

\usetheme{Boadilla}
\usecolortheme{beaver}
\useinnertheme{rectangles}

\setbeamertemplate{navigation symbols}{}

% Font
\usepackage{fontspec}
\setmainfont{Latin Modern Roman}

% Language and typography
\usepackage{polyglossia}
\setdefaultlanguage{english}
\setotherlanguage{french}

\usepackage{csquotes}

\usepackage{microtype}

% Mathematics
\usepackage{mathtools}

% Floats
\usepackage{float}
\usepackage{booktabs}
\usepackage{multicol}

% References
\usepackage{cleveref}


%              %
%%            %%
%%% DOCUMENT %%%
%%            %%
%              %

% Information
\title{Mathematics}
\author[A. Quenon]{Alexandre Quenon}

% Text
\begin{document}
% *** Title page *** %
\begin{frame}
	\titlepage
\end{frame}


% *** Contents *** %
\begin{frame}
	\frametitle{Overview}
	
	\tableofcontents
\end{frame}


% *** Tutorial *** %
%-----
\section{Useful packages}

\begin{frame}
	\frametitle{Packages for mathematics}

	Some packages very useful for mathematics are listed here below:
	\begin{itemize}
		\item the very well-known \emph{amsmath} package which is the backbone for mathematics with \LaTeX{},
		\item \alert{\emph{mathtools}} which is mainly an upgrade of \emph{amsmath},
		\item \emph{cases} which provides the \texttt{numcases} command to number all lines of a system of equations (cf. tutorial C100),
		\item \emph{systeme} which provides command to format a system of equations for better readility (cf. tutorial C100), and
		\item \emph{physics} which provides many commands to facilitate the writing of \enquote{complex} equations including derivatives and partial derivatives (cf. tutorial C102).
	\end{itemize}
\end{frame}


%-----
\section{Writing equations}

\begin{frame}
	\frametitle{Equations}
	\framesubtitle{Main environment for mathematics}
	
	The \structure{main \LaTeX{}'s environment} to write an equation is\dots{} \alert{\texttt{equation}}.
	As an example:
	\begin{equation}
		\vec{\nabla}\cdot\vec{B}=0
		\label{eq::Maxwell:no_magnetic_monopole}
	\end{equation}
	
	The starred version (\texttt{equation*}) disables numbering:
	\begin{equation*}
		\vec{\nabla}\times\vec{E}=-\frac{\partial\vec{B}}{\partial t}
	\end{equation*}
	
	There are also shorter forms:
	\begin{itemize}
		\item the \textbackslash{}[ \ldots \textbackslash{}] wrapper surrounding the equation,
		\item the double \$\$ symbol surrounding the equation (still overly used but it is plain \TeX{} and should not be used).
	\end{itemize}

	\structure{Recommendation}: use \texttt{equation} instead of the short forms:
	\begin{itemize}
		\item it highlights the mathematics in the \LaTeX{} code,
		\item versatility between the numbered and the unnumbered version.
	\end{itemize}
\end{frame}


\begin{frame}
	\frametitle{Equations}
	\framesubtitle{Inline equations -- text-mode \& math-mode}
	
	\structure{Need}: it is sometimes useful to write mathematics inside a text, for instance to describe the variable $\vec{B}$ appearing in \cref{eq::Maxwell:no_magnetic_monopole}.
	To do so, the mathematical formula must be wrapped by single \$ signs.
	
	Inline equations underlines a fundamental behaviour of \LaTeX{}: the difference between \alert{math-mode and text-mode}.
	Compare:
	\begin{itemize}
		\item regular text (text-mode),
		\item $text in mathematical environment$ (math-mode).
	\end{itemize}
	Know the mode inside an environment to understand how \LaTeX{} will behave.
	
	\structure{Recommendation}: use inline equations only
	\begin{itemize}
		\item to express a variable,
		\item for a very short and well-known formula that must not be referred and that do not contain big symbols (integral, sum, etc.).
	\end{itemize}
\end{frame}


%-----
\section{Grouping equations}

\begin{frame}
	\frametitle{Grouping equations}
	\framesubtitle{No alignment inside the group}
	
	\structure{Tool}: \alert{\texttt{gather}} environment, double backslash (\textbackslash\textbackslash) before starting a new equation.
	
	Example with the local equation from Ampere theorem:
	\begin{gather}
		\vec{\nabla} \times \vec{B}
			= \mu_0 \vec{\jmath}
			+ \varepsilon_0 \mu_0 \frac{\partial \vec{E}}{\partial t}, \\
		\intertext{which can be written in the integral form by applying the Green theorem}
		\oint_{C} \vec{B} \cdot \mathrm{d} \vec{l}
			= \mu_0 \iint_S \vec{\jmath} \cdot \mathrm{d} \vec{S}
			+ \varepsilon_0 \mu_0 \iint_S \frac{\partial \vec{E}}{\partial t} \cdot \mathrm{d} \vec{S}.
	\end{gather}
	
	\structure{Extra}: text can be written between equations thanks to the \texttt{intertext} and \texttt{shortintertext} commands.
	\LaTeX{} is in text-mode within these commands and in math-mode within the rest of the \texttt{gather} environment.
\end{frame}

\begin{frame}
	\frametitle{Grouping equations}
	\framesubtitle{Alignment inside the group}
	
	\structure{Tool}: \texttt{align} environment, double backslash (\textbackslash\textbackslash) before starting a new equation, ampersand (\&) to indicate where the alignment is performed.
	
	Examples with the vector potential:
	\begin{align}
		\vec{B} &= \vec{\nabla} \times \vec{A} \\
		\vec{E} &= -\vec{\nabla} V - \frac{\partial \vec{A}}{\partial t}
	\end{align}
	
	 \structure{Extra}: the \texttt{intertext} and \texttt{shortintertext} commands are also available.
	 
	 The alignment is generally performed on the equal sign.
\end{frame}


%-----
\section{Symbols for mathematics}

\begin{frame}
	\frametitle{Common symbols for mathematics}
	
	Mathematics would not be mathematics without any symbols.
	As there is no point to show an exhaustive list of existing mathematical symbols in this tutorial, here follows a list of references:
	\begin{itemize}
		\item your \LaTeX{} editor, which generally provides a list of shortcuts and buttons to generate the correct commands,
		\item a \href{https://fr.sharelatex.com/learn/List_of_Greek_letters_and_math_symbols}{quick review} of symbols from ShareLaTeX,
		\item a \href{http://www.rpi.edu/dept/arc/training/latex/LaTeX_symbols.pdf}{big list} of symbols native from \TeX{} and coming from different packages.
	\end{itemize}
\end{frame}


%-----
%\section{Laying out the mathematics}
%
%\begin{frame}
%	\frametitle{Package options for layout modification}
%
%	It is possible to change the layout of equations thanks to package options:
%	\begin{itemize}
%		\item position of equation numbers
%		\begin{itemize}
%			\item on the right (default) with the \texttt{reqno} option,
%			\item on the left with the \texttt{leqno} option.
%		\end{itemize}
%	\end{itemize}
%\end{frame}


% *** References *** %
\end{document}