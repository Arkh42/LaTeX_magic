%              %
%%            %%
%%% PREAMBLE %%%
%%            %%
%              %

% Document class
\documentclass[11pt]{beamer}

\usetheme{Boadilla}
\usecolortheme{beaver}
\useinnertheme{rectangles}

\setbeamertemplate{navigation symbols}{}

% Font
\usepackage{fontspec}
\setmainfont{Latin Modern Roman}

% Language and typography
\usepackage{polyglossia}
\setdefaultlanguage{english}
\setotherlanguage{french}

\usepackage{csquotes}

\usepackage{microtype}

% Mathematics
\usepackage{mathtools}
\usepackage{cases}
\usepackage{systeme}

% Floats
\usepackage{float}
\usepackage{booktabs}
\usepackage{multicol}

% References
\usepackage{cleveref}


%              %
%%            %%
%%% DOCUMENT %%%
%%            %%
%              %

% Information
\title{Mathematics}
\author[A. Quenon]{Alexandre Quenon}

% Text
\begin{document}
% *** Title page *** %
\begin{frame}
	\titlepage
\end{frame}


% *** Contents *** %
\begin{frame}
	\frametitle{Overview}
	
	\tableofcontents
\end{frame}


% *** Tutorial *** %
%-----
\section{Useful packages}

\begin{frame}
	\frametitle{Packages for mathematics}

	Some packages very useful for mathematics are listed here below:
	\begin{itemize}
		\item \enquote{mathtools} which is mainly an upgrade of the very well-known \enquote{amsmath} package (the backbone for mathematics with \LaTeX{}),
		\item \enquote{cases} which provides the \texttt{numcases} command to number all lines of a system of equations.
	\end{itemize}
\end{frame}


%-----
\section{Writing equations}

\begin{frame}
	\frametitle{Equations}
	
	The main LaTeX environment to write an equation is\dots{} \texttt{equation}.
	As an example:
	\begin{equation}
		\vec{\nabla}\cdot\vec{B}=0
		\label{eq::Maxwell:no_magnetic_monopole}
	\end{equation}
	
	The starred version disables numbering:
	\begin{equation*}
		\vec{\nabla}\times\vec{E}=-\frac{\partial\vec{B}}{\partial t}
	\end{equation*}
	
	There are also shorter forms thanks to:
	\begin{itemize}
		\item the \textbackslash [ \ldots \textbackslash ] wrapper surrounding the equation,
		\item the double \$\$ symbol surrounding the equation (plain \TeX{}, deprecated, should not be used).
	\end{itemize}
	However, I recommend the use of the \texttt{equation} environment because it highlights the mathematics in the \LaTeX{} code and for its versatility between the numbered and the unnumbered version.
\end{frame}


\begin{frame}
	\frametitle{Inline equations}
	
	It is sometimes useful to write mathematics inside a text, for instance to describe the variable $\vec{B}$ appearing in \cref{eq::Maxwell:no_magnetic_monopole}.
	To do so, the mathematical formula must be wrapped by single \$ signs.
	
	Recommendation: try to not abuse of inline equations because they
	\begin{itemize}
		\item can be difficult to read in the text,
		\item could \enquote{ruin} the line space,
		\item cannot be numbered so it is not possible to refer to them.
	\end{itemize}
\end{frame}


%-----
\section{Grouping equations}

\begin{frame}
	\frametitle{Grouping equations}
	\framesubtitle{No alignment inside the group}
	
	Tool: \texttt{gather} environment, double backslash (\textbackslash\textbackslash) before starting a new equation.
	
	Example with the local equation from Ampere theorem:
	\begin{gather}
		\vec{\nabla} \times \vec{B}
			= \mu_0 \vec{\jmath}
			+ \varepsilon_0 \mu_0 \frac{\partial \vec{E}}{\partial t}, \\
		\intertext{which can be written in the integral form by applying the Green theorem}
		\oint_{C} \vec{B} \cdot \mathrm{d} \vec{l}
			= \mu_0 \iint_S \vec{\jmath} \cdot \mathrm{d} \vec{S}
			+ \varepsilon_0 \mu_0 \iint_S \frac{\partial \vec{E}}{\partial t} \cdot \mathrm{d} \vec{S}.
	\end{gather}
	
	Text can be written between equations thanks to the \texttt{intertext} and \texttt{shortintertext} commands.
\end{frame}

\begin{frame}
	\frametitle{Grouping equations}
	\framesubtitle{Alignment inside the group}
	
	Tool: \texttt{align} environment, double backslash (\textbackslash\textbackslash) before starting a new equation, ampersand (\&) to indicate where the alignment is performed.
	
	Examples with the vector potential:
	\begin{align}
		\vec{B} &= \vec{\nabla} \times \vec{A} \\
		\vec{E} &= -\vec{\nabla} V - \frac{\partial \vec{A}}{\partial t}
	\end{align}
	
	 The \texttt{intertext} and \texttt{shortintertext} commands are also available.
	 
	 The alignment is generally performed on the equal sign.
\end{frame}


%-----
\section{Systems of equations}

\begin{frame}
	\frametitle{Systems of equations}
	\framesubtitle{Defined-by-domain functions}
	
	Tool: \texttt{cases} environment, which displays an opening bracket surrounding all equations included in the environment.
	It must be included inside another mathematical equation environment.
	As in \texttt{align}, \textbackslash\textbackslash{} before starting a new line.
	One and only one \& per line can be used to create a column, typically used to specify the domain on which the equation is valid.
	
	A starred version makes the right column \emph{text-mode} instead of \emph{math-mode}.
	A \texttt{dcases} variant makes the environment \emph{displaystyle}.
	
	For examples:
	\begin{equation}
		a = \begin{cases*}
			x^2 + 2					& if  $x<2$  \\
			\int x-3\, \mathrm{d}x	& otherwise
		\end{cases*}
	\end{equation}
	\begin{equation}
		a = \begin{dcases*}
			x^2 + 2					& if  $x<2$  \\
			\int x-3\, \mathrm{d}x	& otherwise
		\end{dcases*}
	\end{equation}
\end{frame}

\begin{frame}
	\frametitle{Systems of equations}
	\framesubtitle{Systems using \texttt{cases}: not very efficient}
	
	The \texttt{cases} environment to write systems:
	\begin{align}
		\begin{cases}
			x  +2y - z  = 1 \\
			x  -3y + 2z = 4 \\
			-x +y  +z   = 0 
		\end{cases}
	\end{align}
	
	Issues:
	\begin{enumerate}
		\item the whole system is numbered but it would be useful to number each line of the system $\rightarrow$ see the \enquote{cases} package,
		\item there is no alignment between the variables like it is sometimes done in algebra.
	\end{enumerate}
\end{frame}

\begin{frame}
	\frametitle{Systems of equations}
	\framesubtitle{Numbering all lines of the system}
	
	Tool: \texttt{numcases} environment from the \enquote{cases} packages:
	\begin{numcases}{}
		x  +2y - z  = 1 \\
		x  -3y + 2z = 4 \\
		-x +y  +z   = 0 
	\end{numcases}
	
	Can also be used for defined-by-domain functions:
	\begin{numcases}{a=}
		x^2 + 2					& if  $x<2$  \\
		\int x-3\, \mathrm{d}x	& otherwise
	\end{numcases}
	
	In addition to the numbering of all lines, it correspond to \texttt{dcases*}, which means that:
	\begin{itemize}
		\item it is directly in \emph{displaystyle},
		\item the right column is in \emph{text-mode}.
	\end{itemize}
\end{frame}

\begin{frame}
	\frametitle{Systems of equations}
	\framesubtitle{Alignment on variables}
	
	Tool: \texttt{systeme} command from the \enquote{systeme} packages.
	Works outside any math environment and inside \texttt{equation}.
	Commas (,) used to separate equations.
	
	Example:
	\begin{equation}
		\systeme{%
			x  +2y - z  = 1,
			x  -3y + 2z = 4,
			-x +y  +z   = 0}
	\end{equation}
	
	Issue: the numbering counter used by the \texttt{systeme} command is independent from the \LaTeX{}'s equation internal counter.
\end{frame}


%-----
\section{Matrices}

\begin{frame}
	\frametitle{Matrices}
	\framesubtitle{Types of matrices}
	
	Matrices can be written by using a \texttt{matrix}-like environment inside a mathematical equation environment such as the ones presented here above.
	
	Several types of matrices exist.
	They differ with the type of delimiters surrounding the matrix:
	\begin{align*}
		\text{\texttt{matrix}} && \text{\texttt{pmatrix}} && \text{\texttt{bmatrix}} && \text{\texttt{Bmatrix}} \\
		%
		\begin{matrix}
			x_{11} & x_{12} \\
			x_{21} & x_{22} \\
		\end{matrix} 
		&&
		\begin{pmatrix}
			x_{11} & x_{12} \\
			x_{21} & x_{22} \\
		\end{pmatrix}
		&&
		\begin{bmatrix}
			x_{11} & x_{12} \\
			x_{21} & x_{22} \\
		\end{bmatrix}
		&&
		\begin{Bmatrix}
			x_{11} & x_{12} \\
			x_{21} & x_{22} \\
		\end{Bmatrix} \\
		%
		%
		&& \text{\texttt{vmatrix}} && \text{\texttt{Vmatrix}} && \\
		%
		&&
		\begin{vmatrix}
			x_{11} & x_{12} \\
			x_{21} & x_{22} \\
		\end{vmatrix}
		&&
		\begin{Vmatrix}
			x_{11} & x_{12} \\
			x_{21} & x_{22} \\
		\end{Vmatrix}
		&& \\
	\end{align*}
\end{frame}

\begin{frame}
	\frametitle{Matrices}
	\framesubtitle{Alignment}
	
	By default, numbers are centred in each column of a matrix:
	\begin{equation*}
		\begin{pmatrix}
			2  & -3 \\
			42 & 0
		\end{pmatrix}
	\end{equation*}
	
	A starred version of each \texttt{matrix} environment offers an optional argument where the alignment can be provided through a letter: \texttt{c} for center, \texttt{r} for right and \texttt{l} for left.
	Example with right alignment:
	\begin{equation*}
		\begin{pmatrix*}[r]
			2  & -3 \\
			42 & 0
		\end{pmatrix*}
	\end{equation*}
\end{frame}


%-----
\section{Laying out the mathematics}

\begin{frame}
	\frametitle{Package options for layout modification}

	It is possible to change the layout of equations thanks to package options:
	\begin{itemize}
		\item position of equation numbers
		\begin{itemize}
			\item on the right (default) with the \texttt{reqno} option,
			\item on the left with the \texttt{leqno} option.
		\end{itemize}
	\end{itemize}
\end{frame}


% *** References *** %
\end{document}