%% Preamble
% ---------

% Document class
\documentclass[11pt]{beamer}

\usetheme{Boadilla}
\usecolortheme{beaver}
\useinnertheme{rectangles}

\setbeamertemplate{navigation symbols}{}

% Font
\usepackage{fontspec}
\setmainfont{Latin Modern Roman}

% Language
\usepackage{polyglossia}
\setdefaultlanguage{english}
\setotherlanguage{french}

\usepackage{csquotes}

% Floats
\usepackage{float}
\usepackage{booktabs}
\usepackage{multicol}


%% Document
% ---------

% Information
\title{Compiling with \LaTeX}
\author[A. Quenon]{Alexandre Quenon}

% Main
\begin{document}
%|---begin


% Title page
\begin{frame}
	\titlepage
\end{frame}


\begin{frame}
	\frametitle{Existing compilers}
	
	Four main compilers:
	\begin{enumerate}
		\item latex,
		\item pdflatex,
		\item lualatex,
		\item xelatex.
	\end{enumerate}
\end{frame}


\begin{frame}
	\frametitle{A very short history}
	
	To understand the reason why several compilers exist and how to choose one, a very short overview of the \TeX/\LaTeX{} history is presented:
	\begin{enumerate}
		\item \TeX{} (Donald Knuth), the original work, generating DVI files;
		\item \LaTeX{} (Leslie Lamport), a layer above \TeX{} which facilitates and standardises the use of \TeX{} (packages, classes, and so on), still generating DVI files;
		\item PDFTeX and PDFLaTeX (Hàn Thế Thành), which generate directly PDF files, allowing to embed several properties such as hyperlinks and metadata;
		\item XeTeX and XeLaTeX, which improve/allow font use and characters management, required for other languages than English;
		\item LuaTeX and LuaLaTeX, which are an attempt to extend the existing \TeX{} with the Lua programming language.
	\end{enumerate}
\end{frame}


\begin{frame}
	\frametitle{\TeX{} vs \LaTeX{}}
	
	Pro-\TeX{}:
	\begin{itemize}
		\item some people really like plain-\TeX{},
		\item from my experience there are no \enquote{good/valid} arguments to do so.
	\end{itemize}

	Pro-\LaTeX{}:
	\begin{itemize}
		\item embed properties in the PDF files, which is important in a more and more digital world;
		\item the last version of \LaTeX (i.e., \LaTeXe{}) is very stable and allows the use of classes and packages which are very useful to automate complex writing generation, such as mathematics.
	\end{itemize}

	There is a third option which will not be discussed: ConTeXt. The philosophy of ConTeXt is to provide a full-access to typography and document's layout to the user, while \LaTeX{}'s mindset consists in automating it through internal rules to let the user focus on the writing.
\end{frame}


\begin{frame}
	\frametitle{Which compiler to choose}
	\framesubtitle{Comparison of the compilers' features}
	
	\begin{table}
		%\caption{Comparison between \LaTeX{} compilers}
		\begin{tabular}{*{5}{c}} \toprule
			&\multicolumn{4}{c}{Compilers} \\ \cmidrule(l){2-5}
			Features	& LaTeX		& PDFLaTeX	& LuaLaTeX	& XeLaTeX	\\ \midrule
			Output		& dvi		& pdf   	& pdf   	& pdf   	\\ 
			Embed prop?	& no		& yes		& yes		& yes 		\\ 
			Fonts   	& MetaFont	& MetaFont	& OpenType	& OpenType 	\\
			Images		& eps		& png/jpg/pdf & png/jpg/pdf & png/jpg/pdf \\
			Compiling	& short		& short		& long		& long		\\ \bottomrule
		\end{tabular}
	\end{table}
\end{frame}


\begin{frame}
	\frametitle{Which compiler to choose}
	\framesubtitle{Discussion built on my own experience}

	Every comment here below is my opinion, even though I provide some arguments:
	\begin{itemize}
		\item the LaTeX compiler should not be used as many documents are published in a \textit{e}-format;
		\item between PDFLaTeX, XeLaTeX and LuaLaTeX, a discussion is possible because
		\begin{itemize}
			\item PDFLaTeX is more stable and the compilation is performed faster;
			\item XeLaTeX is intended for modern font management;
			\item LuaLaTeX is still under development but has been claimed as the successor of PDFLaTeX.
		\end{itemize}
	\end{itemize}

	I used to compile with PDFLaTeX. Now, I tend to compile with LuaLaTeX.
	Whatever your choice is, try to not change it while writing a document because it could affect the packages in use.
\end{frame}


  
% Some references
% https://fr.sharelatex.com/learn/Choosing_a_LaTeX_Compiler
% https://fr.sharelatex.com/blog/2012/12/01/the-tex-family-tree-latex-pdftex-xelatex-luatex-context.html
% https://texfaq.github.io/FAQ-xetex-luatex
% https://www.overleaf.com/blog/500-whats-in-a-name-a-guide-to-the-many-flavours-of-tex
% https://www.gutenberg.eu.org/IMG/pdf/UtiliserLuaLaTeX1_book.pdf
% https://texnique.fr/osqa/questions/643/pdftex-luatex-xetex-lequel-choisir
% http://jobonne.org/luatest1.pdf
% https://www.gutenberg.eu.org/IMG/pdf/UtiliserLuaLaTeX1_book.pdf

%|---end
\end{document}