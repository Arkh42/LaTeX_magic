%% Preamble
% ---------

% Document class
\documentclass[11pt]{beamer}

\usetheme{Boadilla}
\usecolortheme{beaver}
\useinnertheme{rectangles}

\setbeamertemplate{navigation symbols}{}

% Font
\usepackage{fontspec}
\setmainfont{Latin Modern Roman}

% Language
\usepackage{polyglossia}
\setdefaultlanguage{english}
\setotherlanguage{french}


%% Document
% ---------

% Information
\title{Compiling with \LaTeX}
\author[A. Quenon]{Alexandre Quenon}

% Main
\begin{document}
%|---begin

% Title page
\begin{frame}
	\titlepage
\end{frame}

\begin{frame}
	\frametitle{Existing compilers}
	
	Four main compilers:
	\begin{enumerate}
		\item latex,
		\item pdflatex,
		\item lualatex,
		\item xelatex.
	\end{enumerate}
\end{frame}

\begin{frame}
	\frametitle{A very short history}
	
	First steps:
	\begin{enumerate}
		\item \TeX{} (Donald Knuth), the original work;
		\item \LaTeX{} (Leslie Lamport), a layer above \TeX{} which facilitates and standardises the use of \TeX{} (packages, classes, and so on).
	\end{enumerate}
	Both generates DVI files.

	With the evolution of IT, DVI files were slowly replaced by PDF files, which embed several properties such as hyperlinks and metadata: birth of \textit{pdftex} and \textit{pdflatex} (Hàn Thế Thành), which generate PDF files.
	
	Next step was the management of other languages with different characters and support of modern fonts: XeTeX.
	On the other hand, an attempt to extend the existing  \TeX{} has been made with the Lua programming language: LuaTeX.
	Of course, both exist with \LaTeX: XeLaTeX and LuaLaTeX.
\end{frame}

\begin{frame}
	\frametitle{Differences?}
	
	\begin{table}
		\caption{Comparison between \LaTeX{} compilers}
		\begin{tabular}{*{3}{c}}
			Compiler & Output &      Images      \\
			 latex   &  dvi   &       .eps       \\
			pdflatex &  pdf   & .png, .jpg, .pdf \\
			lualatex &  pdf   & .png, .jpg, .pdf \\
			xelatex  &  pdf   & .png, .jpg, .pdf
		\end{tabular}
	\end{table}
\end{frame}

% Some references
% https://fr.sharelatex.com/learn/Choosing_a_LaTeX_Compiler
% https://fr.sharelatex.com/blog/2012/12/01/the-tex-family-tree-latex-pdftex-xelatex-luatex-context.html
% https://texfaq.github.io/FAQ-xetex-luatex
% https://www.overleaf.com/blog/500-whats-in-a-name-a-guide-to-the-many-flavours-of-tex
% https://www.gutenberg.eu.org/IMG/pdf/UtiliserLuaLaTeX1_book.pdf
% https://texnique.fr/osqa/questions/643/pdftex-luatex-xetex-lequel-choisir
% http://jobonne.org/luatest1.pdf
% https://www.gutenberg.eu.org/IMG/pdf/UtiliserLuaLaTeX1_book.pdf

%|---end
\end{document}