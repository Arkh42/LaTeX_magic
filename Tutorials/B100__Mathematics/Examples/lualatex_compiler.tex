%              %
%%            %%
%%% PREAMBLE %%%
%%            %%
%              %

% Document class
\documentclass[11pt, a4paper]{report}

% Font
\usepackage{fontspec}
\setmainfont[%
	SmallCapsFont={* Caps},%	enable small capital font family
	SlantedFont={* Slanted},%	enable slanted font family
]{Latin Modern Roman}

% Language and typography
\usepackage{polyglossia}
\setdefaultlanguage[variant=british]{english}
\setotherlanguage{french}

\usepackage[autostyle=true]{csquotes}

\usepackage{microtype}

% Mathematics
\usepackage{mathtools}

% Hyperref
\usepackage[hidelinks]{hyperref}
\usepackage{cleveref}


%              %
%%            %%
%%% DOCUMENT %%%
%%            %%
%              %

% Information
\title{Mathematics}
\author{Alexandre Quenon}

% Text
\begin{document}
% *** Title page *** %
\maketitle

\tableofcontents


\chapter{The \enquote{mathtools} package}

	Mathematics writing is one of the most advantage of \LaTeX{} compared to common text editors.
	On a first approach, it looks like a programming language but it is in fact quite intuitive.
	
	The first package which was extremely useful in mathematics writing was \texttt{amsmath}.
	Since then, it has been upgraded by the \texttt{mathtools} package that I recommend to use.
	
	
	\section{Writing an equation}
	
		Writing an equation is simply done thanks to the \texttt{equation} environment.
		Maxwell's equations will be used as examples:
		\begin{equation}
			\vec{\nabla}\cdot\vec{B}=0.
			\label{eq::Maxwell:no_magnetic_monopole}
		\end{equation}
		
		
		\subsection{Unnumbered equations}
		
			Automatic numbering can be avoided by using the starred version: \texttt{equation*}.
			For instance:
			\begin{equation*}
				\vec{\nabla}\times\vec{E}=-\frac{\partial\vec{B}}{\partial t}.
			\end{equation*}
			Nevertheless, it is generally recommended to number all equations in scientific documents for easier reference.
			
			Shorter forms of the unnumbered version are offered by the package: the \textbackslash [ \ldots \textbackslash ] wrapper or the double \$\$ symbol.
			Please note that the latter is plain \TeX{}, which means that it should not be used with LaTeX because it is not robust.
			
			Here follow the corresponding examples:
			$$ \iint_{\Sigma_f} \vec{B} \cdot \mathrm{d}\vec{S}, $$
			\[ \oint_{C} \vec{E} \cdot \mathrm{d}\vec{l} = - \iint_{S} \frac{\partial\vec{B}}{\partial t} \cdot \mathrm{d}\vec{S}. \]
			However, I do not recommend to use them. 
			On the one hand, the \texttt{equation*} environment highlights the mathematics when looking in the \LaTeX{} code.
			On the other hand, if the author changes his mind and wants to number the equation, he must simply remove the \texttt{*} character.
		
		
		\subsection{Inline equations}
		
			Inline equations are equations written in the text.
			It can be useful in some circumstances, such as the description of a variable.
			For instance, I could specify that $\vec{B}$ in \cref{eq::Maxwell:no_magnetic_monopole} is the magnetic field.
			
			Unfortunately, inline equations may introduce unpleasant distortion in the text, especially with \enquote{big} symbols.
			As an example, let us use express the acceleration as the derivative of the speed:
			$a=\frac{\mathrm{d}v}{\mathrm{d}t}$.
			It can be seen that the fraction symbol has been compacted to fit with the line space.
			
			It is possible to prevent the fraction from being reshaped by forcing the \textit{math}-mode.
			To do so, the writer must use the \texttt{displaystyle} command, which exists in shorter forms for common mathematical symbols such as fraction.
			Expressing again the acceleration:
			$a=\dfrac{\mathrm{d}v}{\mathrm{d}t}$, or, equivalently,
			$\displaystyle a=\frac{\mathrm{d}v}{\mathrm{d}t}$.
			It can be seen that the space line is increased above and below the line including the equation, creating a somewhat uncomfortable text arrangement.
			
		
		\subsection{General recommendations for mathematics typewriting}
			
			Recommendations:
			\begin{itemize}
				\item use numbered equation only (with the \texttt{equation} environment),
				\item try to avoid inline equations except
				\begin{itemize}
					\item to describe variables or operators,
					\item for very small, less important formulae which do not contain \enquote{big} symbols (e.g., integral, fraction).
				\end{itemize}
			\end{itemize}
%
\end{document}