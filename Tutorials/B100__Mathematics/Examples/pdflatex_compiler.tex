%              %
%%            %%
%%% PREAMBLE %%%
%%            %%
%              %

% Document class
\documentclass[11pt, a4paper, english]{report}

% Font
\usepackage{lmodern}
\usepackage[utf8]{inputenc}
\usepackage[T1]{fontenc}

% Language and typography
\usepackage[main=english, french]{babel}

\usepackage[autostyle=true]{csquotes}

\usepackage[babel=true]{microtype}

% Mathematics
\usepackage{mathtools}

% Hyperref
\usepackage[hidelinks]{hyperref}
\usepackage{cleveref}


%              %
%%            %%
%%% DOCUMENT %%%
%%            %%
%              %

% Information
\title{Mathematics}
\author{Alexandre Quenon}

% Text
\begin{document}
% *** Title page *** %
\maketitle

\tableofcontents


\chapter{The \enquote{mathtools} package}

	Mathematics writing is one of the most advantage of \LaTeX{} compared to common text editors.
	On a first approach, it looks like a programming language but it is in fact quite intuitive.
	
	The first package which was extremely useful in mathematics writing was \texttt{amsmath}.
	Since then, it has been upgraded by the \texttt{mathtools} package that I recommend to use.
	
	
	\section{Writing an equation}
	
		Writing an equation is simply done thanks to the \texttt{equation} environment.
		Maxwell's equations will be used as examples:
		\begin{equation}
			\vec{\nabla}\cdot\vec{B}=0.
			\label{eq::Maxwell:no_magnetic_monopole}
		\end{equation}
		
		Automatic numbering can be avoided by using the starred version: \texttt{equation*}.
		For instance:
		\begin{equation*}
			\vec{\nabla}\times\vec{E}=-\frac{\partial\vec{B}}{\partial t}.
		\end{equation*}
		
		Shorter forms of the unnumbered version are offered by the package: the \textbackslash [ \ldots \textbackslash ] wrapper:
		\[ \oint_{C} \vec{E} \cdot \mathrm{d}\vec{l} = - \iint_{S} \frac{\partial\vec{B}}{\partial t} \cdot \mathrm{d}\vec{S}. \]
		
		Inline equations are equations written in the text.
		For instance, I could specify that $\vec{B}$ in \cref{eq::Maxwell:no_magnetic_monopole} is the magnetic field.
	
	
	\section{Grouping equations}
	
		Two main environments can be used to group equations : \texttt{gather} and \texttt{align}.
		The former groups without aligning, the latter groups and aligns equations.
		
		Example based on \texttt{gather}:
		\begin{gather}
			\vec{\nabla} \times \vec{B}
				= \mu_0 \vec{\jmath}
				+ \varepsilon_0 \mu_0 \frac{\partial \vec{E}}{\partial t}, \\
			\intertext{which can be written in the integral form by applying the Green theorem}
			\oint_{C} \vec{B} \cdot \mathrm{d} \vec{l}
				= \mu_0 \iint_S \vec{\jmath} \cdot \mathrm{d} \vec{S}
				+ \varepsilon_0 \mu_0 \iint_S \frac{\partial \vec{E}}{\partial t} \cdot \mathrm{d} \vec{S}.
		\end{gather}
		
		Example based on \texttt{align}:
		\begin{align}
			\vec{B} &= \vec{\nabla} \times \vec{A}, \\
			\vec{E} &= -\vec{\nabla} V - \frac{\partial \vec{A}}{\partial t}.
		\end{align}
	
	
	\section{Matrices}
	
		Matrices can be generated thanks to the \texttt{matrix} environment which must be used inside a mathematical equation environment.
		There are several variants of \texttt{matrix} which produce different delimiters surrounding the matrix.
		
		The \enquote{mathtools} package offers starred version of the \texttt{matrix} environments which allow to pass an optional argument to specify the alignment inside the matrix's columns.
		
		Example for each type of \texttt{matrix} environments:
		\begin{align*}
			\text{\texttt{matrix}} && \text{\texttt{pmatrix}} && \text{\texttt{bmatrix}} && \text{\texttt{Bmatrix}} \\
			%
			\begin{matrix}
				x_{11} & x_{12} \\
				x_{21} & x_{22} \\
			\end{matrix} 
			&&
			\begin{pmatrix}
				x_{11} & x_{12} \\
				x_{21} & x_{22} \\
			\end{pmatrix}
			&&
			\begin{bmatrix}
				x_{11} & x_{12} \\
				x_{21} & x_{22} \\
			\end{bmatrix}
			&&
			\begin{Bmatrix}
				x_{11} & x_{12} \\
				x_{21} & x_{22} \\
			\end{Bmatrix} \\
			%
			%
			&& \text{\texttt{vmatrix}} && \text{\texttt{Vmatrix}} && \\
			%
			&&
			\begin{vmatrix}
				x_{11} & x_{12} \\
				x_{21} & x_{22} \\
			\end{vmatrix}
			&&
			\begin{Vmatrix}
				x_{11} & x_{12} \\
				x_{21} & x_{22} \\
			\end{Vmatrix}
			&& 
		\end{align*}
		
		Also compare
		\begin{align*}
			\begin{pmatrix}
				2  & -3 \\
				42 & 0
			\end{pmatrix}
			&&
			\text{VS}
			&&
			\begin{pmatrix*}[r]
				2  & -3 \\
				42 & 0
			\end{pmatrix*}
		\end{align*}
%
\end{document}