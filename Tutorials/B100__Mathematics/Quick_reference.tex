%              %
%%            %%
%%% PREAMBLE %%%
%%            %%
%              %

% Document class
\documentclass[11pt]{beamer}

\usetheme{Boadilla}
\usecolortheme{beaver}
\useinnertheme{rectangles}

\setbeamertemplate{navigation symbols}{}

% Font
\usepackage{fontspec}
\setmainfont{Latin Modern Roman}

% Language and typography
\usepackage{polyglossia}
\setdefaultlanguage{english}
\setotherlanguage{french}

\usepackage{csquotes}

\usepackage{microtype}

% Mathematics
\usepackage{mathtools}

% Floats
\usepackage{float}
\usepackage{booktabs}
\usepackage{multicol}

% References
\usepackage{cleveref}


%              %
%%            %%
%%% DOCUMENT %%%
%%            %%
%              %

% Information
\title{Mathematics}
\author[A. Quenon]{Alexandre Quenon}

% Text
\begin{document}
% *** Title page *** %
\begin{frame}
	\titlepage
\end{frame}


% *** Tutorial *** %
\begin{frame}
	\frametitle{Packages for mathematics}

	Some packages very useful for mathematics are listed here below:
	\begin{itemize}
		\item \texttt{mathtools} which is mainly an upgrade of the very well-known \texttt{amsmath} package.
	\end{itemize}
\end{frame}


\begin{frame}
	\frametitle{Equations}
	
	The main LaTeX environment to write an equation is\dots{} \texttt{equation}.
	As an example:
	\begin{equation}
		\vec{\nabla}\cdot\vec{B}=0
		\label{eq::Maxwell:no_magnetic_monopole}
	\end{equation}
	
	The starred version disables numbering:
	\begin{equation*}
		\vec{\nabla}\times\vec{E}=-\frac{\partial\vec{B}}{\partial t}
	\end{equation*}
	
	There are also shorter forms thanks to:
	\begin{itemize}
		\item the \textbackslash [ \ldots \textbackslash ] wrapper surrounding the equation,
		\item the double \$\$ symbol surrounding the equation (plain \TeX{}, deprecated, should not be used).
	\end{itemize}
	However, I recommend the use of the \texttt{equation} environment because it highlights the mathematics in the \LaTeX{} code and for its versatility between the numbered and the unnumbered version.
\end{frame}


\begin{frame}
	\frametitle{Inline equations}
	
	It is sometimes useful to write mathematics inside a text, for instance to describe the variable $\vec{B}$ appearing in \cref{eq::Maxwell:no_magnetic_monopole}.
	To do so, the mathematical formula must be wrapped by single \$ signs.
	
	Recommendation: try to not abuse of inline equations because they
	\begin{itemize}
		\item can be difficult to read in the text,
		\item could \enquote{ruin} the line space,
		\item cannot be numbered so it is not possible to refer to them.
	\end{itemize}
\end{frame}


\begin{frame}
	\frametitle{Package options for layout modification}

	It is possible to change the layout of equations thanks to package options:
	\begin{itemize}
		\item position of equation numbers
		\begin{itemize}
			\item on the right (default) with the \texttt{reqno} option,
			\item on the left with the \texttt{leqno} option.
		\end{itemize}
	\end{itemize}
\end{frame}


% *** References *** %
\end{document}