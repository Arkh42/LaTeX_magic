%             %
%%           %%
%%% CHAPTER %%%
%%           %%
%             %


\chapter{Mathematics -- Basics with the \enquote{mathtools} package}

Mathematics writing is one of the most advantage of \LaTeX{} compared to common text editors.
On a first approach, it looks like a programming language but it is in fact quite intuitive.

The first package which was extremely useful in mathematics writing was \texttt{amsmath}.
Since then, it has been upgraded by the \texttt{mathtools} package that I recommend to use.


\section{Writing an equation}

	Writing an equation is simply done thanks to the \texttt{equation} environment.
	Maxwell's equations will be used as examples.
	So, the 
\begin{lstlisting}[language={[LaTeX]TeX}]
\begin{equation}
	\vec{\nabla}\cdot\vec{B}=0.
	\label{eq::Maxwell:no_magnetic_monopole}
\end{equation}
\end{lstlisting}
	\LaTeX{} code generates
	\begin{equation}
		\vec{\nabla}\cdot\vec{B}=0.
		\label{eq::Maxwell:no_magnetic_monopole}
	\end{equation}
	
	
	\subsection{Unnumbered equations}
	
		Automatic numbering can be avoided by using the starred version: \texttt{equation*}.
		For instance,
\begin{lstlisting}[language={[LaTeX]TeX}]
\begin{equation*}
	\vec{\nabla}\times\vec{E}
		=-\frac{\partial\vec{B}}{\partial t}.
\end{equation*}
\end{lstlisting}	
produces
		\begin{equation*}
			\vec{\nabla}\times\vec{E}
				=-\frac{\partial\vec{B}}{\partial t}.
		\end{equation*}
		Nevertheless, it is generally recommended to number all equations in scientific documents for easier reference.
		
		Shorter forms of the unnumbered version are offered by the package: the \textbackslash [ \ldots \textbackslash ] wrapper or the double \$\$ symbol.
		Please note that the latter is plain \TeX{}, which means that it should not be used with LaTeX because it is not robust.
		
		Here follow the corresponding examples:
\begin{lstlisting}[language={[LaTeX]TeX}]
$$\iint_{\Sigma_f} \vec{B} \cdot \mathrm{d}\vec{S},$$
\end{lstlisting}
		is the code corresponding to
		$$ \iint_{\Sigma_f} \vec{B} \cdot \mathrm{d}\vec{S}, $$
		while
\begin{lstlisting}[language={[LaTeX]TeX}]
\[ \oint_{C} \vec{E} \cdot \mathrm{d}\vec{l}
	= - \iint_{S} \frac{\partial\vec{B}}{\partial t}
	  \cdot \mathrm{d}\vec{S}. \]
\end{lstlisting}
		creates
		\[ \oint_{C} \vec{E} \cdot \mathrm{d}\vec{l}
			= - \iint_{S} \frac{\partial\vec{B}}{\partial t}
			  \cdot \mathrm{d}\vec{S}. \]
		However, I do not recommend to use them. 
		On the one hand, the \texttt{equation*} environment highlights the mathematics when looking in the \LaTeX{} code.
		On the other hand, if the author changes his mind and wants to number the equation, he must simply remove the \texttt{*} character.
	
	
	\subsection{Inline equations}
	
		Inline equations are equations written in the text.
		It can be useful in some circumstances, such as the description of a variable.
		For instance, I could specify that $\vec{B}$ in \cref{eq::Maxwell:no_magnetic_monopole} is the magnetic field.
		To do so, the equation is surrounded by single \$ signs:
\begin{lstlisting}[language={[LaTeX]TeX}]
$\vec{B}$
\end{lstlisting}
		
		Unfortunately, inline equations may introduce unpleasant distortion in the text, especially with \enquote{big} symbols.
		As an example, let us use express the acceleration as the derivative of the speed:
		$a=\frac{\mathrm{d}v}{\mathrm{d}t}$.
		It can be seen that the fraction symbol has been compacted to fit with the line space.
		
		It is possible to prevent the fraction from being reshaped by forcing the \textit{math}-mode.
		To do so, the writer must use the \texttt{displaystyle} command, which exists in shorter forms for common mathematical symbols such as fractions.
		Expressing again the acceleration:
		$a=\dfrac{\mathrm{d}v}{\mathrm{d}t}$, or, equivalently,
		$\displaystyle a=\frac{\mathrm{d}v}{\mathrm{d}t}$.
		It can be seen that the space line is increased above and below the line including the equation, creating a somewhat uncomfortable text arrangement.
		
		The problem presented here above is related to the following \LaTeX{} codes:
\begin{lstlisting}[language={[LaTeX]TeX}]
$a=\frac{\mathrm{d}v}{\mathrm{d}t}$
$a=\dfrac{\mathrm{d}v}{\mathrm{d}t}$
$\displaystyle a=\frac{\mathrm{d}v}{\mathrm{d}t}$
\end{lstlisting}
		
	
	\subsection{General recommendations for equations}
		
		Recommendations:
		\begin{itemize}
			\item use numbered equation only (with the \texttt{equation} environment),
			\item try to avoid inline equations except
			\begin{itemize}
				\item to describe variables or operators,
				\item for very small, less important formulae which do not contain \enquote{big} symbols (e.g., integral, fraction).
			\end{itemize}
		\end{itemize}
	
	
\section{Writing groups of equations}

	Several commands and environments allow to group equations.
	The most-used are presented here after.
	For a complete presentation, please refer to the \enquote{amsmath} and \enquote{mathtools} packages documentation.
	
	
	\subsection{Group of equations}
	
		The first tool which allow to group equations is the \texttt{gather} environment.
		Inside the environment, a double backslash (\textbackslash\textbackslash) indicates the end of an equation.
		Consequently, a new line is produced and another equation can be written.
		Pay attention: no double backslash must be put after the last equation.
		Otherwise, an additional space is added at the end of the group.
		
		Unless the starred version (\texttt{gather*}) is used, all equations are numbered.
		To prevent one line from being numbered, the \texttt{\textbackslash{}notag} or the \texttt{\textbackslash{}nonumber} command can be used.
		
		It is also possible to write text between equations while still being in the \texttt{gather} environment.
		This is done with the \texttt{\textbackslash{}intertext} command, or \texttt{\textbackslash{}shortintertext} which removes extra vertical space.
		It is specifically useful for a mathematical development.
		
		As an example, the	
\begin{lstlisting}[language={[LaTeX]TeX}]
\begin{gather}
	\vec{\nabla} \times \vec{B}
		= \mu_0 \vec{\jmath}
		+ \varepsilon_0 \mu_0 \frac{\partial \vec{E}}{\partial t},
		\\
	\intertext{which can be written in the integral form by 
		applying the Green theorem}
	\oint_{C} \vec{B} \cdot \mathrm{d} \vec{l}
		= \mu_0 \iint_S \vec{\jmath} \cdot \mathrm{d} \vec{S}
		+ \varepsilon_0 \mu_0 \iint_S \frac{\partial \vec{E}}{
			\partial t}	\cdot \mathrm{d} \vec{S}
\end{gather}
\end{lstlisting}	
		code will generate
		\begin{gather}
			\vec{\nabla} \times \vec{B}
				= \mu_0 \vec{\jmath}
				+ \varepsilon_0 \mu_0 \frac{\partial \vec{E}}{\partial t},
				\\
			\intertext{which can be written in the integral form by applying the Green theorem}
			\oint_{C} \vec{B} \cdot \mathrm{d} \vec{l}
				= \mu_0 \iint_S \vec{\jmath} \cdot \mathrm{d} \vec{S}
				+ \varepsilon_0 \mu_0 \iint_S \frac{\partial \vec{E}}{\partial t} \cdot \mathrm{d} \vec{S}
		\end{gather}
		(see the use of \texttt{\textbackslash{}intertext}).
	
	
	\subsection{Group of aligned equations}
	
		The second tool allowing to group equations is the \texttt{align} environment.
		It does the same as the \texttt{gather} environment but also allows to align the equations.
		The alignment is performed thanks to the \& symbol.
		All other commands and symbol performs the same as in the texttt{gather} environment.
		
		For instance,
\begin{lstlisting}[language={[LaTeX]TeX}]
\begin{align}
	\vec{B} &= \vec{\nabla} \times \vec{A} \\
	\vec{E} &= -\vec{\nabla} V - \frac{\partial \vec{A}}{\partial t}.
\end{align}
\end{lstlisting}
		creates
		\begin{align}
			\vec{B} &= \vec{\nabla} \times \vec{A} \\
			\vec{E} &= -\vec{\nabla} V - \frac{\partial \vec{A}}{\partial t}.
		\end{align}
	
	
	\subsection{Using the commands instead of the environments}
	
		The main difference with the \texttt{equation} environment is that all lines can be numbered.


% *** End of chapter ***