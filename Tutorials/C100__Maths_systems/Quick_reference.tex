%              %
%%            %%
%%% PREAMBLE %%%
%%            %%
%              %

% Document class
\documentclass[11pt]{beamer}

\usetheme{Boadilla}
\usecolortheme{beaver}
\useinnertheme{rectangles}

\setbeamertemplate{navigation symbols}{}

% Font
\usepackage{fontspec}
\setmainfont{Latin Modern Roman}

% Language and typography
\usepackage{polyglossia}
\setdefaultlanguage{english}
\setotherlanguage{french}

\usepackage{csquotes}

\usepackage{microtype}

% Mathematics
\usepackage{mathtools}
\usepackage{cases}
\usepackage{systeme}

% Floats
\usepackage{float}
\usepackage{booktabs}
\usepackage{multicol}

% References
\usepackage{cleveref}


%              %
%%            %%
%%% DOCUMENT %%%
%%            %%
%              %

% Information
\title{Mathematics}
\subtitle{Systems of equations}
\author[A. Quenon]{Alexandre Quenon}

% Text
\begin{document}
% *** Title page *** %
\begin{frame}
	\titlepage
\end{frame}


% *** Contents *** %
\begin{frame}
	\frametitle{Overview}
	
	\tableofcontents
\end{frame}


% *** Tutorial *** %
%-----
\section{Useful packages}

\begin{frame}
	\frametitle{Packages for systems of equations}

	Some packages very useful for mathematics are listed here below:
	\begin{itemize}
		\item \emph{mathtools} which is mainly an upgrade of the very well-known \enquote{amsmath} package (the backbone for mathematics with \LaTeX{}),
		\item \emph{cases} which provides the \texttt{numcases} command to number all lines of a system of equations, and
		\item \emph{systeme} which proposes tools to improve the display of the variables of the system.
	\end{itemize}
\end{frame}


%-----
\section{Functions defined by domain}

\begin{frame}
	\frametitle{Functions defined by domain}
	
	\structure{Tool}: \texttt{cases} environment, \textbackslash\textbackslash{} before starting a new line, maximum one \& per line.
	Must be included inside another mathematical equation environment.
	
	For examples:
	\begin{equation}
		a = \begin{cases}
			x^2 + 2					& \text{if} x<2  \\
			\int x-3\, \mathrm{d}x	& \text{if} x \geq 2
		\end{cases}
	\end{equation}
	\begin{equation}
		a = \begin{dcases*}
			x^2 + 2					& if  $x<2$  \\
			\int x-3\, \mathrm{d}x	& otherwise
		\end{dcases*}
	\end{equation}
	
	
	\structure{Extra}: a starred version makes the right column \emph{text-mode} instead of \emph{math-mode}.
	A \texttt{dcases} variant makes the environment \emph{displaystyle}.
	An \texttt{rcases} variant creates a closing bracket on the right side.
\end{frame}


%-----
\section{Systems of equations}

\begin{frame}
	\frametitle{Systems of equations}
	\framesubtitle{One number for the whole system}
	
	\structure{Tool}:  \texttt{cases} environment (same as for defined-by-domain function).
	
	Example:
	\begin{align}
		\begin{cases}
			x  +2y - z  = 1 \\
			x  -3y + 2z = 4 \\
			-x +y  +z   = 0 
		\end{cases}
	\end{align}
	
	
	\structure{Issues}:
	\begin{enumerate}
		\item only one number the whole system but it would be useful to number each line of the system $\rightarrow$ see the \emph{cases} package,
		\item there is no alignment between the variables like it is sometimes done in algebra $\rightarrow$ see the \emph{systeme} package.
	\end{enumerate}
\end{frame}

\begin{frame}
	\frametitle{Systems of equations}
	\framesubtitle{Numbering all lines of the system (1)}
	
	\structure{Tool}: \texttt{numcases} environment from the \emph{cases} package, \textbackslash\textbackslash{} before starting a new line, maximum one \& per line.
	
	Examples:
	\begin{numcases}{}
		x  +2y - z  = 1 \\
		x  -3y + 2z = 4 \\
		-x +y  +z   = 0 
	\end{numcases}
	
	\begin{numcases}{a=}
		x^2 + 2					& if  $x<2$  \\
		\int x-3\, \mathrm{d}x	& otherwise
	\end{numcases}
	
	
	\structure{Extra}: it corresponds to \texttt{dcases*} with all lines numbered, which means that:
	\begin{itemize}
		\item it is directly in \emph{displaystyle},
		\item the right column is in \emph{text-mode}.
	\end{itemize}
\end{frame}

\begin{frame}
	\frametitle{Systems of equations}
	\framesubtitle{Numbering all lines of the system (2)}
	
	\structure{Numbering style}: \texttt{subnumcases} uses the same number and adds a letter in the equation tag.
	
	Example:
	\begin{subnumcases}{}
		x  +2y - z  = 1 \\
		x  -3y + 2z = 4 \\
		-x +y  +z   = 0 
	\end{subnumcases}
\end{frame}

\begin{frame}
	\frametitle{Systems of equations}
	\framesubtitle{Alignment on variables}
	
	\structure{Tool}: \texttt{systeme} command from the \emph{systeme} package, commas (,) used to separate equations.
	Works outside any math environment as well as inside \texttt{equation}.
	
	Example:
	\begin{equation}
		\systeme{%
			x  +2y - z  = 1,
			x  -3y + 2z = 4,
			-x +y  +z   = 0}
	\end{equation}
	
	
	\structure{Issue}: the numbering counter used by the \texttt{systeme} command seems to not work properly the \LaTeX{}'s equation internal counter.
	Consequently, use it inside an \texttt{equation} environment.
\end{frame}




% *** END *** %
\end{document}