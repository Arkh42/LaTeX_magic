%              %
%%            %%
%%% PREAMBLE %%%
%%            %%
%              %

% Document class
\documentclass[11pt]{beamer}

\usetheme{Boadilla}
\usecolortheme{beaver}
\useinnertheme{rectangles}

\setbeamertemplate{navigation symbols}{}

% Font
\usepackage{fontspec}
\setmainfont{Latin Modern Roman}

% Language and typography
\usepackage{polyglossia}
\setdefaultlanguage{english}
\setotherlanguage{french}

\usepackage{csquotes}

\usepackage{microtype}

% Mathematics
\usepackage{mathtools}
\usepackage{physics}

% Floats
\usepackage{float}
\usepackage{booktabs}
\usepackage{multicol}

% References
\usepackage{cleveref}


%              %
%%            %%
%%% DOCUMENT %%%
%%            %%
%              %

% Information
\title{Mathematics}
\subtitle{Matrices}
\author[A. Quenon]{Alexandre Quenon}

% Text
\begin{document}
% *** Title page *** %
\begin{frame}
	\titlepage
\end{frame}


% *** Contents *** %
\begin{frame}
	\frametitle{Overview}
	
	\tableofcontents
\end{frame}


% *** Tutorial *** %
%-----
\section{Useful packages}

\begin{frame}
	\frametitle{Packages for matrices}

	Some packages very useful for mathematics and specifically for matrix computations are listed here below:
	\begin{itemize}
		\item \emph{mathtools} which is mainly an upgrade of the very well-known \emph{amsmath} package (the backbone for mathematics with \LaTeX{}),
		\item \emph{physics} which provides macros to generate easily matrices with specific patterns.
	\end{itemize}
\end{frame}


%-----
\section{Matrices: principle}

\begin{frame}
	\frametitle{Matrices: principle}
	\framesubtitle{Types of matrices}
	
	Matrices can be written by using a \texttt{matrix}-like environment inside a mathematical equation environment such as the ones presented in B100 tutorial.
	
	Several types of matrices exist.
	They differ with the type of delimiters surrounding the matrix:
	\begin{align*}
		\text{\texttt{matrix}} && \text{\texttt{pmatrix}} && \text{\texttt{bmatrix}} && \text{\texttt{Bmatrix}} \\
		%
		\begin{matrix}
			x_{11} & x_{12} \\
			x_{21} & x_{22} \\
		\end{matrix} 
		&&
		\begin{pmatrix}
			x_{11} & x_{12} \\
			x_{21} & x_{22} \\
		\end{pmatrix}
		&&
		\begin{bmatrix}
			x_{11} & x_{12} \\
			x_{21} & x_{22} \\
		\end{bmatrix}
		&&
		\begin{Bmatrix}
			x_{11} & x_{12} \\
			x_{21} & x_{22} \\
		\end{Bmatrix} \\
		%
		%
		&& \text{\texttt{vmatrix}} && \text{\texttt{Vmatrix}} && \\
		%
		&&
		\begin{vmatrix}
			x_{11} & x_{12} \\
			x_{21} & x_{22} \\
		\end{vmatrix}
		&&
		\begin{Vmatrix}
			x_{11} & x_{12} \\
			x_{21} & x_{22} \\
		\end{Vmatrix}
		&& \\
	\end{align*}
\end{frame}

\begin{frame}
	\frametitle{Matrices: principle}
	\framesubtitle{Alignment within the matrix}
	
	By default, numbers are centred in each column of a matrix:
	\begin{equation*}
		\begin{pmatrix}
			2  & -3 \\
			42 & 0
		\end{pmatrix}
	\end{equation*}
	
	A starred version of each \texttt{matrix} environment offers an optional argument where the alignment can be provided through a letter: \texttt{c} for center, \texttt{r} for right and \texttt{l} for left.
	Example with right alignment:
	\begin{equation*}
		\begin{pmatrix*}[r]
			2  & -3 \\
			42 & 0
		\end{pmatrix*}
	\end{equation*}
\end{frame}


%-----
\section{More facilities}

\begin{frame}
	\frametitle{More facilities}
	\framesubtitle{Specific matrices (1)}
	
	\structure{Zero matrix}: \texttt{zeromatrix} or the shorter \texttt{zmat} command.
	
	Examples:
	\begin{align*}
		\begin{pmatrix}
			\zmat{2}
		\end{pmatrix}
		&&
		\begin{bmatrix}
			\zmat{3}
		\end{bmatrix}
		&&
		\begin{vmatrix}
			\zmat{4}
		\end{vmatrix}
	\end{align*}
	

	\structure{Identity matrix}: \texttt{identitymatrix} or the shorter \texttt{imat} command.
	
	Examples:
	\begin{align*}
		\begin{pmatrix}
			\imat{2}
		\end{pmatrix}
		&&
		\begin{bmatrix}
			\imat{3}
		\end{bmatrix}
		&&
		\begin{vmatrix}
			\imat{4}
		\end{vmatrix}
	\end{align*}
\end{frame}


\begin{frame}
	\frametitle{More facilities}
	\framesubtitle{Specific matrices (2)}
	
	\structure{Diagonal matrix}: \texttt{diagonalmatrix} or the shorter \texttt{dmat} command.
	Optional argument to fill spaces.
	
	Examples:
	\begin{align*}
		\begin{pmatrix}
			\dmat{a,b,c}
		\end{pmatrix}
		&&
		\begin{pmatrix}
			\dmat{1,2,3}
		\end{pmatrix}
		&&
		\begin{pmatrix}
			\dmat[0]{1,2,3}
		\end{pmatrix}
	\end{align*}
	
	
	\structure{Automatically filled matrix}: \texttt{xmatrix} or the shorter \texttt{xmat} command.
	The starred version creates automatic indices.
	
	Examples:
	\begin{align*}
		\begin{pmatrix}
			\xmat{1}{2}{3}
		\end{pmatrix}
		&&
		\begin{pmatrix}
			\xmat*{x}{3}{3}
		\end{pmatrix}
		&&
		\begin{pmatrix}
			\xmat*{x}{3}{1}
		\end{pmatrix}
	\end{align*}
\end{frame}


\begin{frame}
	\frametitle{More facilities}
	\framesubtitle{Combinations of patterns}
	
	\structure{Simple way}: use one of the previous commands, then add the other elements above and/or below like in a \enquote{regular} matrix.
	
	Examples:
	\begin{align*}
		\begin{pmatrix}
			\imat{2} \\ a & b
		\end{pmatrix}
		&&
		\begin{pmatrix}
			a & b & c \\
			\zmat{2}{3}
		\end{pmatrix}
	\end{align*}
	
	Issue: impossible to add elements on the right or on the left of a submatrix generated with the \emph{physics}' commands.
	

	\structure{Matrix as a single element}: \texttt{matrixquantity} or the shorter \texttt{mqty} command. 
	
	Example:
	\begin{equation*}
		\begin{pmatrix}
			\mqty{\imat{2}} & \mqty{e\\d} \\
			\mqty{a & b} & c
		\end{pmatrix}
	\end{equation*}
\end{frame}


\end{document}