%              %
%%            %%
%%% PREAMBLE %%%
%%            %%
%              %

% Document class
\documentclass[11pt]{beamer}

\usetheme{Boadilla}
\usecolortheme{beaver}
\useinnertheme{rectangles}

\setbeamertemplate{navigation symbols}{}

% Font
\usepackage{fontspec}
\setmainfont{Latin Modern Roman}

% Language and typography
\usepackage{polyglossia}
\setdefaultlanguage{english}
\setotherlanguage{french}

\usepackage{csquotes}

\usepackage{microtype}

% Mathematics
\usepackage{mathtools}
\usepackage{physics}

% Floats
\usepackage{float}
\usepackage{booktabs}
\usepackage{multicol}

% References
\usepackage{cleveref}


%              %
%%            %%
%%% DOCUMENT %%%
%%            %%
%              %

% Information
\title{Mathematics}
\subtitle{Professional style}
\author[A. Quenon]{Alexandre Quenon}

% Text
\begin{document}
% *** Title page *** %
\begin{frame}
	\titlepage
\end{frame}


% *** Contents *** %
\begin{frame}
	\frametitle{Overview}
	
	\tableofcontents
\end{frame}


% *** Tutorial *** %
%-----
\section{Useful packages}

\begin{frame}
	\frametitle{Packages for professional styling}

	Some packages very useful for mathematics and specifically for matrix computations are listed here below:
	\begin{itemize}
		\item \emph{mathtools} which is mainly an upgrade of the very well-known \emph{amsmath} package (the backbone for mathematics with \LaTeX{}),
		\item \emph{physics} which provides macros to handle efficiently mathematics professional styling.
	\end{itemize}
\end{frame}


%-----
\section{Various macros}

\begin{frame}
	\frametitle{Various macros}
	
	\structure{Absolute value}: \texttt{absolutevalue} command or the shorter form \texttt{abs}.
	Automatic sizing, starred version to cancel it.
	
	Examples:
	\begin{equation*}
		\abs{a}, \quad
		\abs{a^2}, \quad
		\abs{\frac{a}{b}}, \quad
		\abs*{\frac{a}{b}} \, \text{(starred)}
	\end{equation*}
	
	
	\structure{Norm}: \texttt{norm} command.
	Automatic sizing, starred version to cancel it.
	
	Examples:
	\begin{equation*}
		\norm{a}, \quad
		\abs{a^2}, \quad
		\norm{\frac{a}{b}}, \quad
		\norm*{\frac{a}{b}} \, \text{(starred)}
	\end{equation*}
	
	
	\structure{Order}: \texttt{order} command.
	Automatic sizing, starred version to cancel it.
	
	Examples:
	\begin{equation*}
		\order{x}, \quad
		\order{x^2}, \quad
		\order{\frac{1}{x}}, \quad
		\order*{\frac{1}{x}} \, \text{(starred)}
	\end{equation*}
\end{frame}


%-----
\section{Named functions}

\begin{frame}
	\frametitle{Named functions}
	
	Named functions, such as trigonometric (sinus, cosinus, etc.) or logarithmic, must be written in upright font because they are not variables.
	To do so, \emph{mathtools} defines intuitive commands and \emph{physics} extends them.
	For instance, \texttt{sin} for a sinus and \texttt{log} for logarithm, which are actually the true notations.
	An optional argument allows to pass a power to the function.
	
	Compare the wrong and correction notations:
	\begin{itemize}
		\item $sin(x)$ VS $\sin(x)$,
		\item $log(x)$ VS $\log[2](x)$,
		\item $Re(z)$ VS $\Re(z)$ ($\real(z)$ still accepted).
	\end{itemize}
\end{frame}


%-----
\section{Vectors}

\begin{frame}
	\frametitle{Vectors}
	\framesubtitle{Notations}
	
	\structure{Basic style}: to indicate that a variable is a vector and not a scalar, people usually put an arrow above the variable, especially for handwritten documents.
	Tool: \texttt{vec} command.
	
	Example:
	\begin{equation*}
		\vec{a}, \quad
		\norm{\vec{a}}
	\end{equation*}
	

	\structure{Professional style}: professional vector notation consists in putting the variable in bold font.
	Tool: \texttt{vectorbold} command, or the shorter form \texttt{vb}, from the \emph{physics} package.
	
	In addition, unit vectors should be written in bold font with a hat.
	Tool: \texttt{vectorunit}, or the shorter form \texttt{vu}.
	
	Example:
	\begin{equation*}
		\vb{a}, \quad
		\norm{\vb{a}}, \quad
		\vb{a} = a_x \vu{i} + a_y\vu{j} + a_z\vu{k}
	\end{equation*}
\end{frame}

\begin{frame}
	\frametitle{Vectors}
	\framesubtitle{Operators}
	
	\structure{Products}:
	\begin{itemize}
		\item inner/scalar/dot product $\rightarrow$ \texttt{dotproduct}, \texttt{vdot},
		\item cross product $\rightarrow$ \texttt{crossproduct}, \texttt{cross}, \texttt{cp}.
	\end{itemize}
	
	Examples:
	\begin{equation*}
		\vb{a} \vdot \vb{b}, \quad
		\vb{a} \cp \vb{b}
	\end{equation*}
	
	
	\structure{Gradient-based operators}:
	\begin{itemize}
		\item gradient/nabla $\rightarrow$ \texttt{gradient}, \texttt{grad},
		\item divergence $\rightarrow$ \texttt{divergence}, \texttt{div},
		\item curl/rotational $\rightarrow$ \texttt{curl}
		\item Laplace operator/Laplacian $\rightarrow$ \texttt{laplacian}.
	\end{itemize}
	
	Examples:
	\begin{equation*}
		\grad{\Psi}, \quad
		\div, \quad
		\curl, \quad
		\laplacian{\Psi}
	\end{equation*}
\end{frame}


%-----
\section{Differentials and derivatives}

\begin{frame}
	\frametitle{Differentials and derivatives}
	
	\structure{Differential}: the `d' symbol must be in upright font because it is not a variable.
	Tool: \texttt{differential}, or the shorter form \texttt{dd}.
	Optional argument to pass the power.
	
	Examples:
	\begin{equation*}
		\dd, \quad
		\dd{x}, \quad
		\dd[2]{x}, \quad
		\dd(\cos\theta)
	\end{equation*}
	
	
	\structure{Derivative}: same remark as for differential.
	Very long to write with \enquote{regular} \LaTeX{}.
	Tools: \texttt{derivative} or \texttt{dv} for \enquote{normal} derivative and \texttt{partialderivative} or \texttt{pdv} for partial derivative.
	Starred version to make it \emph{inline}.
	
	Examples:
	\begin{equation*}
		\dv{t}, \quad
		\dv{f}{x}, \quad
		\dv{\theta}(\cos[2](\theta)), \quad
		\pdv[n]{f}{x}, \quad
		\pdv{f}{x}{y}
	\end{equation*}
\end{frame}



\end{document}