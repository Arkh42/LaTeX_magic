%              %
%%            %%
%%% PREAMBLE %%%
%%            %%
%              %

% Document class
\documentclass[11pt]{beamer}

\usetheme{Boadilla}
\usecolortheme{beaver}
\useinnertheme{rectangles}

\setbeamertemplate{navigation symbols}{}

% Font
\usepackage{fontspec}
\setmainfont[%
	SmallCapsFont={* Caps},%	enable small capital font family
	SlantedFont={* Slanted},%	enable slanted font family
]{Latin Modern Roman}

% Language
\usepackage{polyglossia}
\setdefaultlanguage{english}
\setotherlanguage{french}

\usepackage[autostyle=true]{csquotes}

% Floats
\usepackage{float}
\usepackage{booktabs}
\usepackage{multicol}


%              %
%%            %%
%%% DOCUMENT %%%
%%            %%
%              %

% Information
\title{Installing \LaTeX}
\subtitle{Distributions and editors}
\author[A. Quenon]{Alexandre Quenon}

% Text
\begin{document}
% *** Title page *** %
\begin{frame}
	\titlepage
\end{frame}


% *** Tutorial *** %
\begin{frame}
	\frametitle{\LaTeX-related software}
	
	To write a document with \LaTeX{}, the user must install specific programs, as for any other text editor.
	These are:
	\begin{itemize}
		\item a distribution,
		\item an editor.
	\end{itemize}

	The \LaTeX{} distribution contains all tools necessary to generate a document (usually a PDF) from the \enquote{code}.
	The editor is a program which simplifies the \LaTeX{} writing by providing shortcuts for compilation (cf. next tutorial) and shortcuts for text styling (e.g., \texttt{CTRL+B} creates bold text).
\end{frame}


\begin{frame}
	\frametitle{Which distribution to choose}
	\framesubtitle{Comparison of LaTeX distributions based on OS compatibility}
	
	\begin{table}
		\begin{tabular}{*{4}{c}} \toprule
			&\multicolumn{3}{c}{Distributions} \\ \cmidrule(l){2-4}
			OS			& \TeX{} Live	& MiK\TeX{}	& Mac\TeX{}	\\ \midrule
			Windows		& 	yes			& 	yes   	& 	pdf   	\\ 
			Linux		& 	yes			& 	yes		& 	yes		\\ 
			MacOS   	& 	no			& 	yes		& 	yes	 	\\ \bottomrule
		\end{tabular}
	\end{table}
\end{frame}

\begin{frame}
	\frametitle{Which distribution to choose}
	\framesubtitle{Feedback from my own experience}
	
	As I have used both Windows and Linux (mainly CentOS) operating systems but never Mac, I cannot provide any feedback about \alert{Mac\TeX{}}. Sorry for Mac users\dots
	
	However, I have used extensively \alert{MiK\TeX{}} (on Windows only) and \alert{\TeX{} Live} (both Linux and Windows):
	\begin{itemize}
		\item I faced problems for using some packages or compilers with MiK\TeX{} that were not existing with \TeX{} Live such as
		\begin{itemize}
			\item compilation issues when using LuaLaTeX (cf. tutorial B002),
			\item \texttt{biber} engine not recognised while I was using the \texttt{biblatex} package,
		\end{itemize}
		\item this is what I experienced
		\begin{itemize}
			\item I compared up-to-date versions,
			\item I did it a few years ago so it could have changed since the comparison was done,
		\end{itemize}
		\item I \textbf{recommend} the use of \textbf{\TeX{} Live}.
	\end{itemize}
\end{frame}


\begin{frame}
	\frametitle{Which editor to choose}
	\framesubtitle{Comparison of LaTeX editors based on OS compatibility}
	
	\begin{table}
		\begin{tabular}{*{4}{c}} \toprule
			&\multicolumn{3}{c}{Distributions} \\ \cmidrule(l){2-4}
			OS			& \TeX{}studio	& \TeX{}maker	& \TeX{}nicCenter	\\ \midrule
			Windows		& 	yes			& 	yes   		& 	yes   	\\ 
			Linux		& 	yes			& 	yes			& 	no		\\ 
			MacOS   	& 	yes			& 	yes			& 	no	 	\\ \bottomrule
		\end{tabular}
	\end{table}
\end{frame}

\begin{frame}
	\frametitle{Which editor to choose}
	\framesubtitle{Feedback from my own experience}
	
	As I have used both \TeX{}maker and \TeX{}studio but \alert{\TeX{}nicCenter}.
	Sorry that I cannot provide any feedback about it\dots
	
	I have used a lot \alert{\TeX{}maker} (on Windows only) then I switched to  \alert{\TeX{}studio} (both Linux and Windows) because to my opinion:
	\begin{itemize}
		\item the tool was more \enquote{clever} considering the key management (references, bibliography),
		\item it proposes automatic alignment in \texttt{tabular} environments (even though this is not working when the \texttt{multicolumn} command is used).
	\end{itemize}
	However, for those who like a fine control, e.g. for compilation, \TeX{}maker is providing a more visible interface.
	
	Anyway, the editor is a matter of taste. So try some and do not hesitate to change and to compare them to find yours.
\end{frame}


\begin{frame}
	\frametitle{Installation process}
	
	%TODO
\end{frame}

  
% *** References *** %
\end{document}