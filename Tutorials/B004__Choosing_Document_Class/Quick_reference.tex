%              %
%%            %%
%%% PREAMBLE %%%
%%            %%
%              %

% Document class
\documentclass[11pt]{beamer}

\usetheme{Boadilla}
\usecolortheme{beaver}
\useinnertheme{rectangles}

\setbeamertemplate{navigation symbols}{}

% Font
\usepackage{fontspec}
\setmainfont{Latin Modern Roman}

% Language and typography
\usepackage{polyglossia}
\setdefaultlanguage{english}
\setotherlanguage{french}

\usepackage{csquotes}

\usepackage{microtype}

% Floats
\usepackage{float}
\usepackage{booktabs}

% Scientific
\usepackage{siunitx}

% References
\usepackage{cleveref}


%              %
%%            %%
%%% DOCUMENT %%%
%%            %%
%              %

% Information
\title{Choosing a \LaTeX{} document class}
\author[A. Quenon]{Alexandre Quenon}

% Text
\begin{document}
% *** Title page *** %
\begin{frame}
	\titlepage
\end{frame}


% *** Contents *** %
\begin{frame}
	\frametitle{Overview}
	
	\tableofcontents
\end{frame}


% *** Tutorial *** %
%-----
\section{What a document class is}

\begin{frame}[fragile]
	\frametitle{What a document class is}
	\framesubtitle{Introduction}
	
	\structure{Command}.
	In the a \LaTeX{} file, the first line of code is generally:
	\verb+\documentclass{class-name}+.
	As the command name states, it corresponds to the choice of a document class.
	
	
	\structure{Effects}.
	The document class has several effects, including:
	\begin{itemize}
		\item the type of document which is generated (report, slides, etc.),
		\item the base layout of the document (level of headings, base font size, margins\ldots),
		\item sometimes an automatic call to specific packages or a limitation in packages that can be loaded.
	\end{itemize}
	We will return to what a \enquote{package} is in tutorial B005.
\end{frame}

\begin{frame}
	\frametitle{What a document class is}
	\framesubtitle{Standard classes}
	
	Any \LaTeX{} distribution embeds document classes which answer to most needs: the so-called \emph{standard classes}.
	
	They are generally sufficient for most people but they can suffer from the drawbacks listed below:
	\begin{itemize}
		\item not well-suited to European standards (e.g., margins),
		\item limited or complicated possibilities of layout modifications,
		\item too \enquote{mainstream} (no, I'm kidding).
	\end{itemize}


	To answer specific needs, the big \LaTeX{} community has proposed many \emph{alternative classes} \cite{CTAN_Class}.
	Nevertheless, the user must be aware that:
	\begin{itemize}
		\item some classes might not be available in all \LaTeX{} distributions,
		\item some classes are marginally used (so is the maintenance).
	\end{itemize}

	Hence this tutorial which does not aim to present an exhaustive list of all existing classes but rather to show some interesting ones.
\end{frame}


%-----
\section{Classes for common documents}

\begin{frame}
	\frametitle{Classes for common documents}
	\framesubtitle{\LaTeX{}'s standard classes: presentation}
	
	The table below presents the main standard classes used for most of the documents that are written:
	\begin{table}[h]
		\begin{tabular}{*{3}{c}}
			\toprule
			 Class  & \multicolumn{2}{c}{Highest heading} \\ \cmidrule{2-3}
			        &      Name       &     Level     \\ \midrule
			article &     section     &       0       \\
			report  &     chapter     &       1       \\
			 book   &      part       &       2       \\ \bottomrule
		\end{tabular}
	\end{table}
	Refer to the \enquote{*\_complier\_\_std\_*} files in the \enquote{Examples} folder to see the difference.
\end{frame}

\begin{frame}[fragile]
	\frametitle{Classes for common documents}
	\framesubtitle{\LaTeX{}'s standard classes: options review (1)}
	
	
	\structure{Paper size}.
	By default, the document is a portrait paper. 
	It is possible to use the \texttt{landscape} option.
	
	Most common options are:
	\begin{table}[h]
		\begin{tabular}{*{3}{c}}
			\toprule
			     Option      & \multicolumn{2}{c}{Physical dimensions} \\
			 \cmidrule{2-3}  &    Height     &          Width          \\ \midrule
			\texttt{a4paper} & \SI{297}{\mm} &      \SI{210}{\mm}      \\
			\texttt{a5paper} & \SI{210}{\mm} &      \SI{148}{\mm}      \\
			\texttt{b5paper} & \SI{250}{\mm} &      \SI{176}{\mm}      \\ \bottomrule
		\end{tabular}
	\end{table}

	The physical dimensions are accessible within the document with the \verb|\paperwidth| and \verb|\paperheight| commands.
	
	
	\vspace*{1ex}
	
	
	\structure{Font size}.
	Three options: \texttt{10pt}, \texttt{11pt}, \texttt{12pt}.
\end{frame}

\begin{frame}
	\frametitle{Classes for common documents}
	\framesubtitle{\LaTeX{}'s standard classes: options review (2)}
	
	
	\structure{Printed sides}.
	Two options: \texttt{oneside} (default for article and report), \texttt{twoside} (default for book).
	
	
	\structure{Chapter starting page}.
	Two options:
	\begin{enumerate}
		\item \texttt{openright} to force chapters to start on a right-hand page by creating an empty left-hand page if necessary,
		\item \texttt{openany} to start chapters on any page.
	\end{enumerate}
	These options are not valid for article as there are no chapters.
	
	
	\vspace*{1ex}
	
	
	\structure{Number of columns}
	Two options: \texttt{onecolumn}, \texttt{twocolumn}.
	
	
	\vspace*{1ex}
	
	
	\structure{Maths-related options}.
	
	By default, equation tags (numbers) are created on the right-hand side.
	This is modified by the \texttt{leqno} option.
	
	By default, displayed math environments (cf. tutorial B100) are centred.
	They can be flushed left with the \texttt{fleqn} option.
\end{frame}




%-----
\section{TO SORT}

\begin{frame}
	\frametitle{Review of main classes}
	\frametitle{Alternate classes for common documents}
	
%	Any \LaTeX{} distribution embeds document classes which answer to most needs: the so-called \emph{standard classes}.
	
	\begin{table}[h]
		\begin{tabular}{*{3}{c}}
			\toprule
			 Class   & Highest heading  & Heading level \\ \midrule
			scrartcl &     section      &       0       \\
			scrreprt &     chapter      &       1       \\
			scrbook  &       part       &       2       \\
			 memoir  &       book       &       3       \\
			 octavo  &     chapter      &       2       \\
			  ncc    & option-dependent &      N/A      \\
			 novel   &       N/A        &      N/A      \\ \bottomrule
		\end{tabular}
	\end{table}
\end{frame}


\begin{frame}
	\frametitle{Review of main classes}
	\frametitle{Alternate classes for other documents}
	
%	Any \LaTeX{} distribution embeds document classes which answer to most needs: the so-called \emph{standard classes}.
	\structure{Paper ands journals}
	\begin{table}[h]
		\begin{tabular}{*{2}{c}}
			\toprule
			  Class    &                  Journal                   \\ \midrule
			elsarticle &             Elsevier journals              \\
			 IEEEtran  & IEEE Transactions journals and conferences \\ \bottomrule
		\end{tabular}
	\end{table}

	\structure{Miscellaneous}
	\begin{table}[h]
		\begin{tabular}{*{2}{c}}
			\toprule
			   Class     &        Purpose        \\ \midrule
			   beamer    & Presentation (slides) \\
			  powerdot   & Presentation (slides) \\
			  moderncv   &   Curriculum Vitae    \\
			modernposter &   Scientific poster   \\
			 tikzposter  &   Scientific poster   \\ \bottomrule
		\end{tabular}
	\end{table}
\end{frame}


% *** References *** %
\section{References}

\begin{frame}
	\frametitle{References}
	
	\begin{thebibliography}{Biblio}
		\bibitem[CTAN]{CTAN_Class}
		Comprehensive \TeX{} Archive Network (CTAN).
		\newblock Class, \url{https://ctan.org/topic/class}.
	\end{thebibliography}
\end{frame}

% *** END *** %
\end{document}