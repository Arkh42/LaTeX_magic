%% Preamble
% ---------

% Document class
\documentclass[11pt]{beamer}

\usetheme{Boadilla}
\usecolortheme{beaver}
\useinnertheme{rectangles}

\setbeamertemplate{navigation symbols}{}

% Font
\usepackage{fontspec}
\setmainfont{Latin Modern Roman}

% Language and typography
\usepackage{polyglossia}
\setdefaultlanguage{english}
\setotherlanguage{french}

\usepackage{csquotes}

\usepackage{microtype}

% Floats
\usepackage{float}
\usepackage{booktabs}
\usepackage{multicol}


%% Document
% ---------

% Information
\title[Font, Language and Typography]{Setting Font, Language and Typography}
\author[A. Quenon]{Alexandre Quenon}

% Text
\begin{document}
%******begin******


% Title page
\begin{frame}
	\titlepage
\end{frame}


\begin{frame}
	\frametitle{Accented letters and font encoding}

	For languages like French, for which accented letters are widely used, the \LaTeX{} user must be cautious:
	\begin{itemize}
		\item if \alert<1>{compiling with \textit{latex} or \textit{pdflatex}},
	\only<1>{%
		\begin{itemize}
			\item the \texttt{inputenc} package must be used to be able to typewrite direcly accented letters from the keyboard,
			\begin{itemize}
				\item with inputenc é and à are correct,
				\item without the package the user must write \textbackslash'\{e\} and \textbackslash\`\ \{a\},
				\item the encoding must be provided as an optional argument (utf8 is encouraged),
			\end{itemize}
			\item the \texttt{fontenc} package to enable the correct hyphenation rules depending on font encoding
			\begin{itemize}
				\item the font type must be provided as an optional argument,
				\item \texttt{T1} is recommended for both English and French,
			\end{itemize}
		\end{itemize}
		}
		\item if \alert<2>{compiling with \textit{xelatex} or \textit{lualatex}},
	\only<2>{%
		\begin{itemize}
			\item the \texttt{fontspec} package must be used to enable the rich font management,
			\begin{itemize}
				\item accented letters can be directly typewritten,
				\item the font can be selected without using another package.
			\end{itemize}
		\end{itemize}
	}
	\end{itemize}
\end{frame}


\begin{frame}
	\frametitle{Font selection}
	
	For clarity and readability, original or artistic fonts should be avoided.
	A common font was \textit{Computer Modern}, modernised by the \textit{Latin Modern} font. However, both have been extensively used, so that some people consider them boring.
	
	Depending on the compiler:
	\begin{itemize}
		\item for \textit{latex} and \textit{pdflatex} compilation,
		\begin{itemize}
			\item selection is made by calling the corresponding package,
			\item \textit{Latin Modern} is loaded by the \texttt{lmodern} package,
		\end{itemize}
		\item for \textit{xelatex} and \textit{lualatex} compilation,
		\begin{itemize}
			\item selection is performed through the \texttt{fontspec} package,
			\item advanced options regarding the ligatures can be passed to the package.
		\end{itemize}
	\end{itemize}
\end{frame}


\begin{frame}
	\frametitle{Selecting the language}
	
	There are two efficient packages for selecting a language and load the typographical rules of the language:
	\begin{enumerate}
		\item \texttt{babel}, which works with LaTeX, LuaLaTeX and XeLaTeX;
		\item \texttt{polyglossia}, which was espacially made for XeLaTeX but also works for LuaLaTeX.
	\end{enumerate}
	
	Both packages work with the compilers related to the LaTeX engine described above. Pay attention that \texttt{polyglossia} relies on the \texttt{fontspec} package, so it does not compile with \textit{latex} nor \textit{pdflatex}.
\end{frame}


\begin{frame}
	\frametitle{Multilingual documents}
	
	It is possible to create multilingual documents:
	\begin{itemize}
		\item with \texttt{babel}
		\begin{itemize}
			\item languages are loaded with the package options,
			\item the main language is specified with the \texttt{main=} identifiers in the options,
			\item the user can locally change the language with the \texttt{selectlanguage} command or the \texttt{otherlanguage} environment,
		\end{itemize}
		\item with \texttt{polyglossia}
		\begin{itemize}
			\item languages are loaded with specific commands in the preamble,
			\item the user can locally change the language with the \texttt{(lang)} environment (e.g. \texttt{french}),
			\item \texttt{babel} commands are also available for compatibility.
		\end{itemize}
	\end{itemize}

	Refer to the examples to see the use of commands and environments, depending on the compiler and the choice between \texttt{babel}/\texttt{polyglossia}.
\end{frame}


\begin{frame}
	\frametitle{Quotations}
	
	A powerful package for quoting text is \texttt{csquotes}.
	It provides:
	\begin{itemize}
		\item the \texttt{enquote} command to quote a text according to the defined language,
		\item the \texttt{foreignquote} command which combines \texttt{enquote} and \texttt{foreignlanguage} from babel/polyglossia,
		\item the \texttt{textquote} command for formal quoting (i.e. with reference citation),
		\item the \texttt{textcquote} for formal quoting interfaced with bibliography management packages\footnote{Concerned packages are natbib, jurabib and biblatex.}.
	\end{itemize}

	The package is originally intended for working in conjunction with \texttt{babel}.
	However, it generally works with \texttt{polyglossia}, even though I faced strange behaviours with XeLaTeX in French.
\end{frame}


\begin{frame}
	\frametitle{Enhanced documents with micro-typography}
	
	Micro-typography is a set of techniques which allow to enhance the aspect of a text by reducing the potential sources of disturbance for the reader.
	It includes:
	\begin{itemize}
		\item character protrusion (a.k.a. margin kerning), which consists in adjusting the characters at the margins of a typeset text for optical alignment;
		\item font expansion which is the method to use a wider or narrower variant of a font to make interword spacing more even;
		\item tracking (letterspacing).
	\end{itemize}

	The effects of \texttt{microtype} are compiler-dependent.
	In summary, the compilers which make the highest benefits from it are PDFLaTeX and LuaLaTeX.
\end{frame}


\begin{frame}
	\frametitle{Packages order}
	
	Summary of the packages which must be loaded, observing the following order:
	\begin{itemize}
		\item for \textit{latex} and \textit{pdflatex} compilation,
		\begin{enumerate}
			\item lmodern (or any other package relative to font selection),
			\item inputenc,
			\item fontenc,
			\item babel,
			\item csquotes,
			\item microtype,
		\end{enumerate}
		\item for \textit{xelatex} and \textit{lualatex} compilation,
		\begin{enumerate}
			\item fontspec,
			\item polyglossia (or babel),
			\item csquotes,
			\item microtype.
		\end{enumerate}
	\end{itemize}
\end{frame}


  
% Some references

%******end******
\end{document}