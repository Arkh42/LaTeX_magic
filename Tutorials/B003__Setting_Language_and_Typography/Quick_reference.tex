%% Preamble
% ---------

% Document class
\documentclass[11pt]{beamer}

\usetheme{Boadilla}
\usecolortheme{beaver}
\useinnertheme{rectangles}

\setbeamertemplate{navigation symbols}{}

% Font
\usepackage{fontspec}
\setmainfont{Latin Modern Roman}

% Language and typography
\usepackage{polyglossia}
\setdefaultlanguage{english}
\setotherlanguage{french}

\usepackage{csquotes}

% Floats
\usepackage{float}
\usepackage{booktabs}
\usepackage{multicol}


%% Document
% ---------

% Information
\title[Font, Language and Typography]{Setting Font, Language and Typography}
\author[A. Quenon]{Alexandre Quenon}

% Text
\begin{document}
%******begin******


% Title page
\begin{frame}
	\titlepage
\end{frame}


\begin{frame}
	\frametitle{Selecting the language}
	
	
\end{frame}

\begin{frame}
	\frametitle{Accented letters and font encoding}

	For languages like French, for which accented letters are widely used, the \LaTeX{} user must be cautious:
	\begin{itemize}
		\item if \alert<1>{compiling with \textit{latex} or \textit{pdflatex}},
	\only<1>{%
		\begin{itemize}
			\item the \texttt{inputenc} package must be used to be able to typewrite direcly accented letters from the keyboard,
			\begin{itemize}
				\item with inputenc é and à are correct,
				\item without the package the user must write \textbackslash'\{e\} and \textbackslash\`\ \{a\},
				\item the encoding must be provided as an optional argument (utf8 is encouraged),
			\end{itemize}
			\item the \texttt{fontenc} package to enable the correct hyphenation rules depending on font encoding
			\begin{itemize}
				\item the font type must be provided as an optional argument,
				\item \texttt{T1} is recommended for both English and French,
			\end{itemize}
		\end{itemize}
		}
		\item if \alert<2>{compiling with \textit{xelatex} or \textit{lualatex}},
	\only<2>{%
		\begin{itemize}
			\item the \texttt{fontspec} package must be used to enable the rich font management,
			\begin{itemize}
				\item accented letters can be directly typewritten,
				\item the font can be selected without using another package.
			\end{itemize}
		\end{itemize}
	}
	\end{itemize}
\end{frame}


\begin{frame}
	\frametitle{Font selection}
	
	For clarity and readability, original or artistic fonts should be avoided.
	A common font was \textit{Computer Modern}, modernised by the \textit{Latin Modern} font. However, both have been extensively used, so that some people consider them boring.
	
	Depending on the compiler:
	\begin{itemize}
		\item for \textit{latex} and \textit{pdflatex} compilation,
		\begin{itemize}
			\item selection is made by calling the corresponding package,
			\item \textit{Latin Modern} is loaded by the \texttt{lmodern} package,
		\end{itemize}
		\item for \textit{xelatex} and \textit{lualatex} compilation,
		\begin{itemize}
			\item selection is performed through the \texttt{fontspec} package,
			\item advanced options regarding the ligatures can be passed to the package.
		\end{itemize}
	\end{itemize}
\end{frame}


\begin{frame}
	\frametitle{Packages order}
	
	Summary of the packages which must be loaded, observing the following order:
	\begin{itemize}
		\item for \textit{latex} and \textit{pdflatex} compilation,
		\begin{enumerate}
			\item lmodern (or any other package relative to font selection),
			\item inputenc,
			\item fontenc,
		\end{enumerate}
		\item for \textit{xelatex} and \textit{lualatex} compilation,
		\begin{enumerate}
			\item fontspec.
		\end{enumerate}
	\end{itemize}
\end{frame}


  
% Some references

%******end******
\end{document}