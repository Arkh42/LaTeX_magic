%% Preamble
% ---------

% Document class
\documentclass[11pt, a4paper, english]{report}

% Font
\usepackage{lmodern}
\usepackage[utf8]{inputenc}
\usepackage[T1]{fontenc}

% Language and typography
\usepackage[main=english, french]{babel}

\usepackage[autostyle=true]{csquotes}

\usepackage[babel=true]{microtype}

% Hyperref
\usepackage[hidelinks]{hyperref}

% Lorem Ipsum (for test purpose)
\usepackage{lipsum}


%% Document
% ---------

% Information
\title{Setting Font, Language and Typography}
\author{Alexandre Quenon}

% Text
\begin{document}
%******begin******

\maketitle

\tableofcontents


\chapter{Font selection examples}
	
	This chapter provides a quick overview of the available commands to modify the text font locally:
	\begin{itemize}
		\item normal,
		\item slanted (rarely used),
		\item italic,
		\item bold,
		\item italic bold,
		\item small capitals.
	\end{itemize}
	
	All examples are based on the Latin Modern Roman font family.
	Text is generated thanks to the well-known \textit{Lorem Ipsum}.
	
	
	\section{Normal font example}
	
	\lipsum[1]
	
	
	\section{Slanted font example}
	
	\textsl{\lipsum[1]}
	
	
	\section{Italic font example}
	
	\textit{\lipsum[1]}
	
	
	\section{Bold font example}
	
	\textbf{\lipsum[1]}
	
	
	\section{Italic bold font example}
	
	\textbf{\textit{\lipsum[1]}}
	
	
	\section{Small capital font example}
	
	\textsc{\lipsum[1]}
	

\chapter{Language selection and typography}

	
	\section{Some English}
	
		A list in English, with some typographical rules which must be observed:
		\begin{itemize}
			\item item 1,
			\item item 2, and
			\item item $n$.
		\end{itemize}
	
		In the previous example, one can see that the first paragraph is not indented; in addition, there are no white spaces before typographical symbols such as colons and semi-colons.
		
		Here is a use of the \texttt{csquotes} package in conjunction with \enquote{babel}. The English quotes are used. 


	%\selectlanguage{french}%		Command
	\begin{otherlanguage}{french}%	Environment
	\section{Un peu de français}
	
		Une liste en français, comprenant les règles typographiques qui doivent être suivie:
		\begin{itemize}
			\item item 1;
			\item item 2;
			\item item $n$.
		\end{itemize}
		
		Contrairement à une idée reçue très répandue, il n'y a pas d'interligne entre les paragraphes en français; par ailleurs, en français, une espace fine se place devant les symboles typographiques tels que les points-virgules.
		
		Remarque importante: contrairement à l'anglais, tous les paragraphes sont indentés en français; malheureusement, le language principal de ce document étant l'anglais, le package \textit{babel} ne respecte pas l'ensemble des règles typographiques.
		
		Voici un exemple d'utilisation du package \texttt{csquotes} en conjonction avec \enquote{babel}. L'usage des guillemets français, et des espaces, est respecté.
	\end{otherlanguage}
		

	
	
%******end******
\end{document}