%% Preamble
% ---------

% Document class
\documentclass[11pt, a4paper]{report}

% Font
\usepackage{fontspec}
\setmainfont[%
	SmallCapsFont={* Caps},%	enable small capital font family
	SlantedFont={* Slanted},%	enable slanted font family
]{Latin Modern Roman}

% Language and typography
\usepackage{polyglossia}
\setdefaultlanguage[variant=british]{english}
\setotherlanguage{french}

\usepackage[autostyle=true]{csquotes}

\usepackage{microtype}

% Hyperref
\usepackage[hidelinks]{hyperref}

% Lorem Ipsum (for test purpose)
\usepackage{lipsum}


%% Document
% ---------

% Information
\title{Setting Font, Language and Typography}
\author{Alexandre Quenon}

% Text
\begin{document}
%******begin******

\maketitle

\tableofcontents


\chapter{Font selection examples}

	This chapter provides a quick overview of the available commands to modify the text font locally:
	\begin{itemize}
		\item normal,
		\item slanted (rarely used),
		\item italic,
		\item bold,
		\item italic bold,
		\item small capitals.
	\end{itemize}

	All examples are based on the Latin Modern Roman font family.
	Text is generated thanks to the well-known \textit{Lorem Ipsum}.


	\section{Normal font example}
	
		\lipsum[1]
		
	
	\section{Slanted font example}
	
		\textsl{\lipsum[1]}
	
	
	\section{Italic font example}
	
		\textit{\lipsum[1]}
		
	
	\section{Bold font example}
	
		\textbf{\lipsum[1]}
	
	
	\section{Italic bold font example}
	
		\textbf{\textit{\lipsum[1]}}
		
	
	\section{Small capital font example}
	
		\textsc{\lipsum[1]}
	

\chapter{Language selection and typography}

	A list in English, with some typographical rules which must be observed:
	\begin{itemize}
		\item item 1,
		\item item 2, and
		\item item $n$.
	\end{itemize}
	
	In the previous example, one can see that the first paragraph is not indented; in addition, there are no white spaces before typographical symbols such as colons and semi-colons.
	
	Here is a use of the \texttt{csquotes} package in conjunction with \enquote{babel}. The English quotes are used. 
	
	
	%\selectlanguage{french}%		Command
	\begin{french}%	Environment
		\section{Un peu de français}
		
		Une liste en français, comprenant les règles typographiques qui doivent être suivie:
		\begin{itemize}
			\item item 1;
			\item item 2;
			\item item $n$.
		\end{itemize}
		
		Contrairement à une idée reçue très répandue, il n'y a pas d'interligne entre les paragraphes en français; par ailleurs, en français, une espace fine se place devant les symboles typographiques tels que les points-virgules.
		
		Remarque importante: \textit{polyglossia} semble légèrement plus performant que \textit{babel} puisque l'indentation du premier paragraphe est respectée. Par contre, aucun des deux ne respectent l'utilisation de tirets.
		
		Voici un exemple d'utilisation du package \texttt{csquotes} en conjonction avec \enquote{polyglossia}. L'usage des guillemets français, et des espaces, est respecté.
	\end{french}

%******end******
\end{document}